\documentclass[english]{../thermomemo/thermomemo}

\usepackage{amsmath, amsthm, amssymb}
\usepackage[english]{babel}
\usepackage[T1]{fontenc}
\usepackage{graphicx}
\usepackage{mathtools}
\usepackage[utf8]{inputenc}
\usepackage{pgf}
\usepackage{tikz}
\usepackage{url}
\usepackage{hyperref}
\usepackage{enumerate}
\usepackage[font=small,labelfont=bf]{caption}
\usepackage{booktabs}
\usepackage{multicol}
\usepackage{siunitx}
\usepackage[version=4]{mhchem}
\usepackage{xcolor}
\hypersetup{
  colorlinks,
  linkcolor={red!50!black},
  citecolor={blue!50!black},
  urlcolor={blue!80!black}
}

\newcommand{\mathsec}[1]{\texorpdfstring{#1}{TEXT}} % Command to put math symbols in section headers. Use like \(...)section{\mathsec{$\<symbol>$}}
\newcommand{\code}[1]{\colorbox{lightgray}{#1}} % For pretty inline code formatting
\DeclareSIUnit{\calorie}{cal}

\title{User guide}
\author{Vegard Gjeldvik Jervell, Morten Hammer}
\date{\today}
\begin{document}
\frontmatter
\tableofcontents

\section{Introduction}
This document is intended for generic user documentation. Also see
\url{https://github.com/thermotools/thermopack/wiki}.

\section{Phase keys}
The phase keys are defined in \path{src/thermopack_constants.f90},
and are shown in Table~\ref{tab:phase_flags_thermopack}.
\begin{table}[ht!]
  \centering
  \begin{tabular}{l c l}
    \toprule
    Phase & Key & Description\\
    \midrule
    Two-phase & 0 & Liquid-vapor two-phase mixture (Code: TWOPH)\\
    Liquid & 1 & Single phase liquid (Code: LIQPH) \\
    Vapor & 2 & Single phase vapor (Code: VAPPH) \\
    Minimum Gibbs & 3 & Single phase root with the minimum Gibbs free energy \\ & & (Code: MINGIBBSPH) \\
    Single & 4 & Single phase not identified as liquid or vapor \\ & & (Code: SINGLEPH) \\
    Solid & 5 & Single phase solid (Code: SOLIDPH) \\
    Fake & 6 & In rare cases no physical roots exist, and a fake liquid root is \\ & & returned (Code: FAKEPH) \\
    \bottomrule
  \end{tabular}
  \caption{Phase flags in thermopack.}
  \label{tab:phase_flags_thermopack}
\end{table}

\section{Cubic Equations of State}

\begin{table}[ht!]
  \centering
  \begin{tabular}{l c }
    \toprule
    Name & Key \\
    \midrule
    Van der Waal & VdW\\
    Soave Redlich Kwong & SRK\\
    Peng Robinson & PR\\
    Schmidt-Wensel & SW\\
    Patel Teja & PT\\
    Translated consistent PR & tcPR\\
    \bottomrule
  \end{tabular}
  \caption{Cubic Equations of state implemented in ThermoPack and the corresponding keys used for initialization.}
  \label{tab:EoS_thermopack}
\end{table}

\subsection{Pure fluid \mathsec{$\alpha$}}
\begin{table}[ht!]
  \centering
  \begin{tabular}{l l}
    \toprule
    Model & Key \\
    \midrule
    Model default$^*$ & Classic\\
    Twu-Coon-Bluck-Cunninghan & TWU\\
    Mathias-Copeman & MC\\
    \href{https://pubs.acs.org/doi/abs/10.1021/i260068a009}{Graboski and Daubert} & GD\\
    \href{https://doi.org/10.1021/cr60137a013}{Redlich-Kwong} & RK\\
    Soave & Soave\\
    \href{https://doi.org/10.1021/i160057a011}{Peng Robinson 76} & PR\\
    \href{https://doi.org/10.1021/ie049580p}{UMR $\alpha$ formulation} & UMR\\
    Peng Robinson 78 & PR78\\
    Van der Waal & VdW\\
    Schmidt-Wensel & SW\\
    Patel Teja & PT\\
    \bottomrule
    \multicolumn{2}{l}{$^*$Will use original $\alpha$ for specified EOS.}\\
    \multicolumn{2}{l}{ E.g. SRK will use Soave $\alpha$,}\\
    \multicolumn{2}{l}{ Peng-Robinson will use PR $\alpha$ etc.}\\
  \end{tabular}
\end{table}

\subsection{\mathsec{$\alpha$} mixing Rules}
\begin{table}[ht!]
  \centering
  \begin{tabular}{l c}
    \toprule
    Name & Key \\
    \midrule
    Van der Waals & Classic or vdW\\
    \href{https://github.com/thermotools/thermopack/tree/main/doc/memo/WongSandler/wongsandler.pdf}{Wong Sandler} & WS \\
    Huron Vidal & HV or HV2 \\
    NRTL & NRTL \\
    \href{https://github.com/thermotools/thermopack/tree/main/doc/memo/UNIFAC/unifac.pdf}{UNIFAC} & UNIFAC \\
    \bottomrule
  \end{tabular}
  \caption{Mixing rules and phases available in thermopack, with the corresponding keys used to identify them.}
  \label{tab:mixing_rules_thermopack}
\end{table}


\section{Adding new fluids}
The fluid database consists of a set of \code{.json}-files in the \code{fluids} directory. These files are are used to auto-generate the FORTRAN-files \code{compdatadb.f90} and \code{saftvrmie\_datadb.f90} by running the respective python scripts \code{compdata.py} and \code{saftvrmie.py} found in the directory \code{addon/pyUtils/datadb/}. The files are generated in the current working directory and must be copied to the \code{src}-directory before recompiling ThermoPack to make the fluids available. 

A \code{<fluid>.json} file must contain a minimal set of data to be valid. This includes the critical point, accentric factor, mole weight and ideal gas heat capacity. 

\subsection{Ideal gas heat capacity}
Several different correlations for the heat capacity are available, selected by the ''correlation''-key in the ''ideal-heat-capacity-1'' field of the fluid files. These are summarized in Table \ref{tab:heat_capacity_correlations}.

\begin{table}[htb]
    \centering
    \caption{Ideal gas heat capacity correlations, and the corresponding keys used in the fluid-database.}
    \begin{tabular}{c l l c}
    \toprule
        Key & Correlation & Equation & Unit \\
    \midrule
        1 & Sherwood, Reid \& Prausnitz$^{(a)}$ & $A + BT + CT^2 + DT^3$ & \si{\calorie\per\gram\per\mol\per\kelvin}\\
        2 & API-Project & 44 & -\\
        3 & Hypothetic components & - & -\\
        4 & Sherwood, Reid \& Prausniz$^{(b)}$ & $A + BT + CT^2 + DT^3$ & \si{\joule\per\mol\per\kelvin}\\
        5 & Ici (Krister Strøm) & $A + BT + CT^2 + DT^3 + ET^{-2}$ & \si{\joule\per\gram\per\kelvin}\\
        6 & Chen, Bender (Petter Nekså) & $A + BT + CT^2 + DT^3 + ET^4$ & \si{\joule\per\gram\per\kelvin}\\
        7 & Aiche, Daubert \& Danner$^{(c)}$ & $A + B\left[\frac{C}{T} \sinh{\left(\frac{C}{T}\right)}\right]^2 + D\left[\frac{E}{T} \cosh{\left(\frac{E}{T}\right)}\right]^2$ & \si{\joule\per\kilo\mol\per\kelvin}\\
        8 & Poling, Prausnitz \& O'Connel$^{(d)}$ & R $\left(A + BT + CT^2 + DT^3 + ET^4\right)$ & \si{\joule\per\mol\per\kelvin}\\
        9 & Linear function and fraction & $A + BT + \frac{C}{T + D}$ & \si{\joule\per\mol\per\kelvin}\\
        10 & Leachman \& Valenta for \ce{H2} & - & -\\
        11 & Use TREND model & - & -\\
        12 & Shomate equation$^*$ & $A + BT_s + CT_s^2 + DT_s^3 + ET_s^{-2}$ & \si{\joule\per\mol\per\kelvin}\\
    \bottomrule
    \multicolumn{2}{l}{$^{(a)}$3rd ed. $^{(c)}$DIPPR-database} & \\
    \multicolumn{2}{l}{$^{(b)}$4th ed. $^{(d)}$5th ed.} & \\
    \multicolumn{2}{l}{$^*$Note: $T_s = 10^{-3} T$}
    \end{tabular}
    \label{tab:heat_capacity_correlations}
\end{table}

\end{document}