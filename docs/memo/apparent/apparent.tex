\documentclass[english]{../thermomemo/thermomemo}
\usepackage[utf8]{inputenc}

\usepackage{amsmath}
%\input{mathdef}
\usepackage[per-mode=symbol]{siunitx}
\usepackage[numbers]{natbib}
\usepackage{amsmath}
\usepackage{amssymb}
\usepackage{array}% improves tabular environment.
\usepackage{dcolumn}% also improves tabular environment, with decimal centring.
\usepackage{chemformula}     % For easy typing of chemical formulae
\usepackage{booktabs}
\usepackage{a4wide}
\usepackage{xspace}
\usepackage{todonotes}
\presetkeys{todonotes}{inline}{}
\usepackage{subcaption,caption}
%\pdfminorversion=4
\usepackage{tikz}
\usetikzlibrary{arrows}
\usetikzlibrary{snakes}
\usepackage{verbatim}
\usepackage{hyperref}
\hypersetup{
  colorlinks=true,
  linkcolor=blue,
  urlcolor=blue,
  citecolor=blue
}
\usepackage{blkarray, bigstrut}

%
% Egendefinerte
%
% Kolonnetyper for array.sty:
\newcolumntype{C}{>{$}c<{$}}% for å slippe å taste inn disse $
\newcolumntype{L}{>{$}l<{$}}% for å slippe å taste inn disse $
%
\newcommand*{\uunit}[1]{\ensuremath{\mathrm{#1}}}
%\newcommand*{\od}[3][]{\frac{\mathrm{d}^{#1}#2}{\mathrm{d}{#3}^{#1}}}% ordinary derivative
\newcommand*{\od}[3][]{\frac{\dif^{#1}#2}{\dif{#3}^{#1}}}% ordinary derivative
\newcommand*{\pd}[3][]{\frac{\partial^{#1}#2}{\partial{#3}^{#1}}}% partial derivative
\newcommand*{\pdc}[3]{\frac{\partial^{2}#1}{\partial{#2}\partial{#3}}}% partial derivative
\newcommand*{\pdt}[3][]{{\partial^{#1}#2}/{\partial{#3}^{#1}}}% partial
                                % derivative for inline use.
\newcommand{\pone}[3]{\frac{\partial #1}{\partial #2}_{#3}}% partial
                                % derivative with information of
                                % constant variables
\newcommand{\ponel}[3]{\frac{\partial #1}{\partial #2}\bigg|_{#3}} % partial derivative with information of constant variable. A line is added.
\newcommand{\ptwo}[3]{\frac{\partial^{2} #1}{\partial #2 \partial
    #3}} % partial differential in two different variables
\newcommand{\pdn}[3]{\frac{\partial^{#1}#2}{\partial{#3}^{#1}}}% partial derivative
\newcommand*{\pder}[2]{\left(\frac{\partial #1}{\partial #2}\right)}
\newcommand*{\pdder}[2]{\left(\frac{\partial^2 #1}{\partial #2^2}\right)}
\newcommand*{\pdersub}[3]{\left(\frac{\partial #1}{\partial #2}\right)_{#3}}

% Total derivative:
\newcommand*{\ttd}[2]{\frac{\mathrm{D} #1}{\mathrm{D} #2}}
\newcommand*{\td}[2]{\frac{\mathrm{d} #1}{\mathrm{d} #2}}
\newcommand*{\ddt}{\frac{\partial}{\partial t}}
\newcommand*{\ddx}{\frac{\partial}{\partial x}}
% Vectors etc:
% For Computer Modern:

\DeclareMathAlphabet{\mathsfsl}{OT1}{cmss}{m}{sl}
\renewcommand*{\vec}[1]{\boldsymbol{#1}}%
\newcommand*{\vektor}[1]{\boldsymbol{#1}}%
\newcommand*{\tensor}[1]{\mathsfsl{#1}}% 2. order tensor
\newcommand*{\matr}[1]{\tensor{#1}}% matrix
\renewcommand*{\div}{\boldsymbol{\nabla\cdot}}% divergence
\newcommand*{\grad}{\boldsymbol{\nabla}}% gradient
% fancy differential from Claudio Beccari, TUGboat:
% adjusts spacing automatically
\makeatletter
\newcommand*{\dif}{\@ifnextchar^{\DIfF}{\DIfF^{}}}
\def\DIfF^#1{\mathop{\mathrm{\mathstrut d}}\nolimits^{#1}\gobblesp@ce}
\def\gobblesp@ce{\futurelet\diffarg\opsp@ce}
\def\opsp@ce{%
  \let\DiffSpace\!%
  \ifx\diffarg(%
    \let\DiffSpace\relax
  \else
    \ifx\diffarg[%
      \let\DiffSpace\relax
    \else
      \ifx\diffarg\{%
        \let\DiffSpace\relax
      \fi\fi\fi\DiffSpace}
\makeatother
%
\newcommand*{\me}{\mathrm{e}}% e is not a variable (2.718281828...)
%\newcommand*{\mi}{\mathrm{i}}%  nor i (\sqrt{-1})
\newcommand*{\mpi}{\uppi}% nor pi (3.141592...) (works for for Lucida)
%
% lav tekst-indeks/subscript/pedex
\newcommand*{\ped}[1]{\ensuremath{_{\text{#1}}}}
% høy tekst-indeks/superscript/apex
\newcommand*{\ap}[1]{\ensuremath{^{\text{#1}}}}
\newcommand*{\apr}[1]{\ensuremath{^{\mathrm{#1}}}}
\newcommand*{\pedr}[1]{\ensuremath{_{\mathrm{#1}}}}
%
\newcommand*{\volfrac}{\alpha}% volume fraction
\newcommand*{\surften}{\sigma}% coeff. of surface tension
\newcommand*{\curv}{\kappa}% curvature
\newcommand*{\ls}{\phi}% level-set function
\newcommand*{\ep}{\Phi}% electric potential
\newcommand*{\perm}{\varepsilon}% electric permittivity
\newcommand*{\visc}{\mu}% molecular (dymamic) viscosity
\newcommand*{\kvisc}{\nu}% kinematic viscosity
\newcommand*{\cfl}{C}% CFL number

\newcommand*{\cons}{\vec U}
\newcommand*{\flux}{\vec F}
\newcommand*{\dens}{\rho}
\newcommand*{\svol}{\ensuremath v}
\newcommand*{\temp}{\ensuremath T}
\newcommand*{\vel}{\ensuremath u}
\newcommand*{\mom}{\dens\vel}
\newcommand*{\toten}{\ensuremath E}
\newcommand*{\inten}{\ensuremath e}
\newcommand*{\press}{\ensuremath p}
\renewcommand*{\ss}{\ensuremath a}
\newcommand*{\jac}{\matr A}
%
\newcommand*{\abs}[1]{\lvert#1\rvert}
\newcommand*{\bigabs}[1]{\bigl\lvert#1\bigr\rvert}
\newcommand*{\biggabs}[1]{\biggl\lvert#1\biggr\rvert}
\newcommand*{\norm}[1]{\lVert#1\rVert}
%
\newcommand*{\e}[1]{\times 10^{#1}}
\newcommand*{\ex}[1]{\times 10^{#1}}%shorthand -- for use e.g. in tables
\newcommand*{\exi}[1]{10^{#1}}%shorthand -- for use e.g. in tables
\newcommand*{\nondim}[1]{\ensuremath{\mathit{#1}}}% italic iflg. ISO. (???)
\newcommand*{\rey}{\nondim{Re}}
\newcommand*{\acro}[1]{\textsc{\MakeLowercase{#1}}}%acronyms etc.

\newcommand{\nto}{\ensuremath{\mbox{N}_{\mbox{\scriptsize 2}}}}
\newcommand{\chfire}{\ensuremath{\mbox{CH}_{\mbox{\scriptsize 4}}}}
%\newcommand*{\checked}{\ding{51}}
\newcommand{\coto}{\ensuremath{\text{CO}_{\text{\scriptsize 2}}}}
\newcommand{\celsius}{\ensuremath{^\circ\text{C}}}
\newcommand{\subl}{\ensuremath{\text{sub}}}
\newcommand{\spec}{\text{spec}}
\newcommand{\sat}{\text{sat}}
\newcommand{\sol}{\text{sol}}
\newcommand{\liq}{\text{liq}}
\newcommand{\vap}{\text{vap}}
\newcommand{\amb}{\text{amb}}
\newcommand{\tr}{\text{tr}}
\newcommand{\crit}{\text{crit}}
\newcommand{\entr}{\ensuremath{\text{s}}}
\newcommand{\fus}{\text{fus}}
\newcommand{\flash}[1]{\ensuremath{#1\text{-flash}}}
\newcommand{\spce}[2]{\ensuremath{#1\, #2\text{ space}}}
\newcommand{\spanwagner}{\text{Span--Wagner}}
\newcommand{\triplepoint}{\text{TP triple point}}
\newcommand{\wrpt}{\text{with respect to~}}
\newcommand{\tpd}{\ensuremath{\text{tpd}}\xspace}
\newcommand{\TPD}{\ensuremath{\text{TPD}}\xspace}
\newcommand{\lp}{\ensuremath{\left(}\xspace}
\newcommand{\rp}{\ensuremath{\right)}\xspace}
\newcommand{\mbn}[0]{\mathbf n}
\newcommand{\mbe}[0]{\mathbf e}
\newcommand{\mbx}[0]{\mathbf x}
\newcommand{\app}{\ensuremath{\text{App}}\xspace}
\newcommand{\nacl}{\ensuremath{\text{\ch{NaCl}}}\xspace}
\newcommand{\na}{\ensuremath{\text{\ch{Na+}}}\xspace}
\newcommand{\cl}{\ensuremath{\text{\ch{Cl-}}}\xspace}
\newcommand{\ideal}{\ensuremath{\text{Id}}\xspace}

\title{Apparent composition approach}
\author{Morten Hammer}

\graphicspath{{gfx/}}

\begin{document}
\frontmatter
\tableofcontents
\section{Introduction}
\citet[Chap. 13, Sec. 7]{Michelsen2007} describe the apparent
composition approach in the context of chemical
reactions. \citet[App. C.3]{Mogensen2014} uses this approach when
modelling phase equilibrium for electrolyte system, using the salts as
apparent composition.

The idea is to use a model with all species ($\vektor{n}$) to model
other species, the apparent species ($\vektor{e}$). The relation
between the apparent and real species are derived from the following,
\begin{equation}
  A\lp T, V, \mbe \rp = A\lp T, V, \mbn \rp,
\label{eq:helmholtz}
\end{equation}
and its differentials. We immediately see that the volume and
temperature differentials are unaffected by the mole number
representation.

Using salt (\ch{NaCl}) to represent ions (\ch{Na+},\ch{Cl-}) as an
example, some additional differentials are required. The apparent mole
vector will then be $\mbe$ (salts) while the actual mole vector will
be $\mbn$ (ions). We then have,

\begin{align}
  F\lp T, V, \mbn \rp &= F\lp T, V, \mbe \rp,\\
  F_i^\app = \pdersub{F}{e_i}{T,V} &=
  \sum_k\pdersub{F}{n_k}{T,V}\pder{n_k}{e_i}
  = \sum_k v_{ik} \pdersub{F}{n_k}{T,V},\\
  F_{ij}^\app &
  = \sum_k v_{ik} \sum_m v_{jm} F_{ij}.
\end{align}
Here $v_{ik}$ is the stoichiometric composition of ion $j$ in salt
$i$. For a system of \ch{NaCl}, \ch{H2O} and \ch{CO2}, $\mathbf{v}$
will look as follows,
\begin{equation}
  \begin{blockarray}{*{4}{c} l}
    \begin{block}{*{4}{>{$\footnotesize}c<{$}} l}
      \ch{H2O} & \ch{CO2} & \ch{Na+} & \ch{Cl-} & \\
    \end{block}
    \begin{block}{[*{4}{c}]>{$\footnotesize}l<{$}}
      1 & 0 & 0 & 0  \bigstrut[t] & \ch{H2O} \\
      0 & 1 & 0 & 0 & \ch{CO2} \\
      0 & 0 & 1 & 1 & \ch{NaCl} \\
    \end{block}
  \end{blockarray}
\end{equation}

With this approach, no special handling of the charge balance in the
phases are required when calculating phase equilibrium. Another
advantage is equilibrium calculation of solid \ch{NaCl}, as the
solid fugacity only need to equate the fugacity of the fictitious
dissolved salt component.

\section{Thermodynamic differentials}

If we consider the chemical potentials (enthalpy, entropy, Gibbs free
energy, Helmholtz free energy or internal energy) as a function of the
real or apparent composition, the value remains the same. IE. it is
only the compositional differentials of the potentials that differ
when introducing the apparent composition.

We therefore need to map only the fugacities, compressibillity factor
and their differentials.

\subsection{Compressibillity factor}
For the compressibillity factor, the following applies,
\begin{equation}
  \frac{PV}{RT} = \sum_j e_j Z^\app = \sum_i n_i Z.
  \label{eq:compressibillity_factor_relation}
\end{equation}
We therefore have
\begin{equation}
  Z^\app = Z \frac{\sum_i n_i}{\sum_j e_j }.
  \label{eq:compressibillity_factor}
\end{equation}
The pressure and temperature differential must be scaled in the same
manner for the apparent mode, while the compositional differentials
are slightly more complicated.

\subsection{Fugacity coefficients}
\begin{equation}
  \ln \lp x_\nacl^\app \varphi_\nacl^\app \rp  = v_\na \ln \lp x_\na \varphi_\na \rp + v_\cl \ln \lp x_\cl \varphi_\cl \rp
\label{eq:fugacity_coeff}
\end{equation}

Assuming we have one mole of water, and want to look at the infinite
dilution value of $\ln \varphi_\nacl^\app$. The mole number relation between \nacl and the ions \na and \cl are one-to-one, giving,
\begin{gather}
  \ln n_\nacl + \ln \varphi_\nacl^\app - \ln \lp 1 + n_\nacl \rp  = 2 \ln n_\nacl + \ln \lp\varphi_\na\varphi_\cl\rp - 2\ln \lp 1 + 2 n_\nacl \rp \nonumber \\
  \ln \varphi_\nacl^\app = \ln n_\nacl + \ln \lp\varphi_\na\varphi_\cl\rp - 2\ln \lp 1 + 2 n_\nacl \rp + \ln \lp 1 + n_\nacl \rp \nonumber \\
  \lim_{n_\nacl \rightarrow 0}\ln \varphi_\nacl^\app = \lim_{n_\nacl \rightarrow 0} \ln \lp n_\nacl \varphi_\na\varphi_\cl\rp \rightarrow - \infty
\label{eq:fugacity_coeff_inf}
\end{gather}

Since the fugacity depend on the composition, and the infinite
dilution fugacity approaches an infinity value, this approach can be
challenging to handle numerically.

\subsubsection{Differentials}
Additionally there is an effect of the difference in overall mole
numbers, as $\sum_i n_i \neq \sum_i e_i$. See Maribo--Mogensen
\cite[App. C.3]{Mogensen2014}.

The equilibrium condition becomes,
\begin{equation}
  \ln \lp \frac{x^\app \varphi^\app P}{P^0} \rp  = \sum_j v_{ij} \ln \lp \frac{x_j \varphi_j P}{P^0} \rp
\label{eq:fugacity}
\end{equation}

To describe the apparent fugacity coefficient, the relation become,
\begin{equation}
  \ln \varphi^\app_i  = \sum_j v_{ij} \ln x_j \varphi_j - \ln x_i^\app - \lp 1 - \sum_j v_{ij} \rp \ln \frac{P}{P^0}
\label{eq:fugacity_coeff_app}
\end{equation}
Differentiating with respect to temperature,
\begin{equation}
  \pd{\ln \varphi^\app_i}{T}  = \sum_j v_{ij} \ln \pd{\varphi_j}{T}.
\label{eq:fugacity_coeff_app_T}
\end{equation}
Differentiating with respect to pressure,
\begin{equation}
  \pd{\ln \varphi^\app_i}{P}  = \sum_j v_{ij} \ln \pd{\varphi_j}{P} - \frac{1}{P} + \frac{\sum_j v_{ij}}{P}.
\label{eq:fugacity_coeff_app_P}
\end{equation}

Differentiating with respect to mole numbers, using $\sum_i e_i = e_T$,
$n_T = \sum_j \sum_i v_{ij} e_i = n_T$, and

\begin{align}
  x_j  &= \frac{\sum_l v_{jl} e_l}{n_T}\\
  \pd{\ln x_j}{e_k} &= \sum_l \lp\pd{n_l}{e_k}\rp \lp \pd{\ln x_i}{n_l} \rp = \sum_l v_{kl} \lp \pd{\ln n_j}{n_l} -\pd{\ln n_T}{n_l} \rp \nonumber \\
  &= \sum_l v_{kl} \lp \frac{\delta_{jl}}{n_l} - \frac{1}{n_T} \rp = \frac{v_{kj}}{n_j} - \frac{\sum_l v_{kl}}{n_T},
\label{eq:x_ion}
\end{align}

\begin{align}
  \pd{\ln \varphi^\app_i}{e_k}  &= - \frac{\delta_{ik}}{e_k} + \frac{1}{e_T} + \sum_j v_{ij} \pd{\ln \varphi_j}{e_k} + \sum_j v_{ij} \pd{\ln x_j}{e_k} \nonumber \\
  &= - \frac{\delta_{ik}}{e_k} + \frac{1}{e_T} + \sum_j v_{ij} \sum_l v_{kl} \pd{\ln \varphi_j}{n_l} + \sum_j v_{ij} \lp \frac{v_{kj}}{n_j} - \frac{\sum_l v_{kl}}{n_T} \rp
\label{eq:fugacity_coeff_app_n}
\end{align}

\subsection{Ideal properties}
In order to calculate equilibrium between a salt (solid), and the
apparent composition in the fluid phases, the ideal Gibbs free energy
is required.

For the enthalpy we have,
\begin{equation}
  h^\ideal \lp \vektor{e},  T \rp = \sum_j e_j h_j^\ideal \lp T \rp,
\label{eq:id_enthalpy}
\end{equation}
and for the entropy,
\begin{equation}
  s^\ideal \lp \vektor{e},  T, P \rp = \sum_j e_j \lp s_j^{\ideal,*} \lp T \rp  - R \ln \frac{e_j}{e_T} - R \ln P \rp.
\label{eq:id_entropy}
\end{equation}
Here $s_j^{\ideal,*} \lp T \rp$ simply is the temperature integral of $Cp(T)/T$.

The ideal Gibbs free energy is,
\begin{equation}
  g^\ideal \lp \vektor{e},  T, P \rp = h^\ideal \lp \vektor{e},  T \rp - T s^\ideal \lp \vektor{e},  T, P \rp.
\label{eq:id_gibbs}
\end{equation}

The ideal chemical potential therefore becomes,
\begin{align}
  \mu_i &= \pd{g^\ideal}{e_i} = h^\ideal_i - T \lp s_i^{\ideal,*}  - R \ln \frac{e_i}{e_T} - R \ln P \rp, \\
  &= h^\ideal_i - T s_i^{\ideal,*}  + RT \ln \lp x_i P \rp.
\label{eq:id_chempot}
\end{align}

Since the $\ln \lp x_i P \rp$ is accounted for in the fugacity
coefficient, we only need to account for $h^\ideal_i - T
s_i^{\ideal,*}$.

\clearpage
\bibliographystyle{plainnat}
\bibliography{../thermopack}

\end{document}
