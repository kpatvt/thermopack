\documentclass[english]{../thermomemo/thermomemo}
\pdfminorversion=4
\usepackage[utf8]{inputenc}

%\usepackage{amsmath}
%\input{mathdef}
\usepackage[per-mode=symbol]{siunitx}
\usepackage[numbers]{natbib}
\usepackage{amsmath}
\usepackage{amssymb}
%\usepackage{mathtools}
\usepackage{array}% improves tabular environment.
\usepackage{dcolumn}% also improves tabular environment, with decimal centring.

\usepackage{booktabs}
\usepackage{a4wide}
\usepackage{xspace}
\usepackage{mhchem}
\usepackage{siunitx}
\DeclareSIUnit{\atmosphere}{atm}
\usepackage{todonotes}
\presetkeys{todonotes}{inline}{}
\usepackage{subcaption,caption}
%\pdfminorversion=4
\usepackage{tikz}
\usetikzlibrary{arrows}
\usetikzlibrary{snakes}
\usepackage{verbatim}
\usepackage{hyperref}
\hypersetup{
  colorlinks=true,
  linkcolor=blue,
  urlcolor=blue,
  citecolor=blue
}
%
% Egendefinerte
%
% Kolonnetyper for array.sty:
\newcolumntype{C}{>{$}c<{$}}% for å slippe å taste inn disse $
\newcolumntype{L}{>{$}l<{$}}% for å slippe å taste inn disse $
%
%\newcommand*{\od}[3][]{\frac{\mathrm{d}^{#1}#2}{\mathrm{d}{#3}^{#1}}}% ordinary derivative
\newcommand*{\od}[3][]{\frac{\dif^{#1}#2}{\dif{#3}^{#1}}}% ordinary derivative
\newcommand*{\pd}[3][]{\frac{\partial^{#1}#2}{\partial{#3}^{#1}}}% partial derivative
\newcommand*{\pdc}[3]{\frac{\partial^{2}#1}{\partial{#2}\partial{#3}}}% partial derivative
\newcommand*{\pdcc}[3]{\frac{\partial^{3}#1}{\partial{#2}^{2}\partial{#3}}}% partial derivative
\newcommand*{\pdth}[4]{\frac{\partial^{3}#1}{\partial{#2}\partial{#3}\partial{#4}}}% partial derivative
\newcommand*{\pdt}[3][]{{\partial^{#1}#2}/{\partial{#3}^{#1}}}% partial
                                % derivative for inline use.
\newcommand{\pone}[3]{\frac{\partial #1}{\partial #2}_{#3}}% partial
                                % derivative with information of
                                % constant variables
\newcommand{\ponel}[3]{\frac{\partial #1}{\partial #2}\bigg|_{#3}} % partial derivative with informatio of constant variable. A line is added.
\newcommand{\ptwo}[3]{\frac{\partial^{2} #1}{\partial #2 \partial
    #3}} % partial differential in two different variables
\newcommand{\pdn}[3]{\frac{\partial^{#1}#2}{\partial{#3}^{#1}}}% partial derivative

% Total derivative:
\newcommand*{\ttd}[2]{\frac{\mathrm{D} #1}{\mathrm{D} #2}}
\newcommand*{\td}[2]{\frac{\mathrm{d} #1}{\mathrm{d} #2}}
\newcommand*{\ddt}{\frac{\partial}{\partial t}}
\newcommand*{\ddx}{\frac{\partial}{\partial x}}
% Vectors etc:
% For Computer Modern:

\DeclareMathAlphabet{\mathsfsl}{OT1}{cmss}{m}{sl}
\renewcommand*{\vec}[1]{\boldsymbol{#1}}%
\newcommand*{\vektor}[1]{\boldsymbol{#1}}%
\newcommand*{\tensor}[1]{\mathsfsl{#1}}% 2. order tensor
\newcommand*{\matr}[1]{\tensor{#1}}% matrix
\renewcommand*{\div}{\boldsymbol{\nabla\cdot}}% divergence
\newcommand*{\grad}{\boldsymbol{\nabla}}% gradient
% fancy differential from Claudio Beccari, TUGboat:
% adjusts spacing automatically
\makeatletter
\newcommand*{\dif}{\@ifnextchar^{\DIfF}{\DIfF^{}}}
\def\DIfF^#1{\mathop{\mathrm{\mathstrut d}}\nolimits^{#1}\gobblesp@ce}
\def\gobblesp@ce{\futurelet\diffarg\opsp@ce}
\def\opsp@ce{%
  \let\DiffSpace\!%
  \ifx\diffarg(%
    \let\DiffSpace\relax
  \else
    \ifx\diffarg[%
      \let\DiffSpace\relax
    \else
      \ifx\diffarg\{%
        \let\DiffSpace\relax
      \fi\fi\fi\DiffSpace}
\makeatother
%
\newcommand*{\me}{\mathrm{e}}% e is not a variable (2.718281828...)
%\newcommand*{\mi}{\mathrm{i}}%  nor i (\sqrt{-1})
\newcommand*{\mpi}{\uppi}% nor pi (3.141592...) (works for for Lucida)
%
% lav tekst-indeks/subscript/pedex
\newcommand*{\ped}[1]{\ensuremath{_{\text{#1}}}}
% høy tekst-indeks/superscript/apex
\newcommand*{\ap}[1]{\ensuremath{^{\text{#1}}}}
\newcommand*{\apr}[1]{\ensuremath{^{\mathrm{#1}}}}
\newcommand*{\pedr}[1]{\ensuremath{_{\mathrm{#1}}}}
%
\newcommand*{\volfrac}{\alpha}% volume fraction
\newcommand*{\surften}{\sigma}% coeff. of surface tension
\newcommand*{\curv}{\kappa}% curvature
\newcommand*{\ls}{\phi}% level-set function
\newcommand*{\ep}{\Phi}% electric potential
\newcommand*{\perm}{\varepsilon}% electric permittivity
\newcommand*{\visc}{\mu}% molecular (dymamic) viscosity
\newcommand*{\kvisc}{\nu}% kinematic viscosity
\newcommand*{\cfl}{C}% CFL number

\newcommand*{\cons}{\vec U}
\newcommand*{\flux}{\vec F}
\newcommand*{\dens}{\rho}
\newcommand*{\svol}{\ensuremath v}
\newcommand*{\temp}{\ensuremath T}
\newcommand*{\vel}{\ensuremath u}
\newcommand*{\mom}{\dens\vel}
\newcommand*{\toten}{\ensuremath E}
\newcommand*{\inten}{\ensuremath e}
\newcommand*{\press}{\ensuremath p}
\renewcommand*{\ss}{\ensuremath a}
\newcommand*{\jac}{\matr A}
%
\newcommand*{\lb}{\left(}
\newcommand*{\rb}{\right)}
\newcommand*{\abs}[1]{\lvert#1\rvert}
\newcommand*{\bigabs}[1]{\bigl\lvert#1\bigr\rvert}
\newcommand*{\biggabs}[1]{\biggl\lvert#1\biggr\rvert}
\newcommand*{\norm}[1]{\lVert#1\rVert}
%
\newcommand*{\e}[1]{\times 10^{#1}}
\newcommand*{\ex}[1]{\times 10^{#1}}%shorthand -- for use e.g. in tables
\newcommand*{\exi}[1]{10^{#1}}%shorthand -- for use e.g. in tables
\newcommand*{\nondim}[1]{\ensuremath{\mathit{#1}}}% italic iflg. ISO. (???)
\newcommand*{\rey}{\nondim{Re}}
\newcommand*{\acro}[1]{\textsc{\MakeLowercase{#1}}}%acronyms etc.

\newcommand{\nto}{\ensuremath{\mbox{N}_{\mbox{\scriptsize 2}}}}
\newcommand{\chfire}{\ensuremath{\mbox{CH}_{\mbox{\scriptsize 4}}}}
%\newcommand*{\checked}{\ding{51}}
\newcommand{\coto}{\ensuremath{\text{CO}_{\text{\scriptsize 2}}}}
\newcommand{\celsius}{\ensuremath{^\circ\text{C}}}
%\newcommand{\clap}{Clapeyron~}
\newcommand{\subl}{\ensuremath{\text{sub}}}
\newcommand{\spec}{\text{spec}}
\newcommand{\sat}{\text{sat}}
\newcommand{\sol}{\text{sol}}
\newcommand{\liq}{\text{liq}}
\newcommand{\vap}{\text{g}}
\newcommand{\amb}{\text{amb}}
\newcommand{\tr}{\text{tr}}
\newcommand{\mie}{\text{Mie}\xspace}
\newcommand{\mca}{\text{MCA}\xspace}
\newcommand{\hs}{\text{HS}\xspace}
\newcommand{\crit}{\text{crit}}
\newcommand{\entr}{\ensuremath{\text{s}}}
\newcommand{\fus}{\text{fus}}
\newcommand{\id}{\ensuremath{\text{id}}\xspace}
\newcommand{\mono}{\ensuremath{\text{mono}}\xspace}
\newcommand{\ms}{\ensuremath{\text{m}}\xspace}
\newcommand{\chain}{\ensuremath{\text{chain}}\xspace}
\newcommand{\assoc}{\ensuremath{\text{assoc}}\xspace}
\newcommand{\seg}{\ensuremath{\text{s}}\xspace}
\newcommand{\flash}[1]{\ensuremath{#1\text{-flash}}}
\newcommand{\spce}[2]{\ensuremath{#1\, #2\text{ space}}}
\newcommand{\spanwagner}{\text{Span--Wagner}}
\newcommand{\triplepoint}{\text{TP triple point}}
\newcommand{\wrpt}{\text{with respect to~}}
\newcommand{\tpd}{\ensuremath{\text{tpd}}\xspace}
\newcommand{\TPD}{\ensuremath{\text{TPD}}\xspace}
\newcommand{\minimum}{\ensuremath{\text{min}}\xspace}
\newcommand{\eq}{\ensuremath{\text{eq}}\xspace}
\newcommand{\lp}{\left(}
\newcommand{\rp}{\right)}
\newcommand{\Lagr}{\ensuremath{\mathcal{L}}\xspace}
\newcommand{\mbn}{\ensuremath{\mathbf{n}}\xspace}
\newcommand{\mbA}{\ensuremath{\mathbf{A}}\xspace}
\newcommand{\mbb}{\ensuremath{\mathbf{b}}\xspace}
\newcommand{\mbB}{\ensuremath{\mathbf{B}}\xspace}
\newcommand{\mbM}{\ensuremath{\mathbf{M}}\xspace}
\newcommand{\mbg}{\ensuremath{\mathbf{g}}\xspace}
\newcommand{\mbd}{\ensuremath{\mathbf{d}}\xspace}
\newcommand{\mbe}{\ensuremath{\mathbf{e}}\xspace}
\newcommand{\one}{\ensuremath{\mathbf{1}}\xspace}
\newcommand{\zero}{\ensuremath{\mathbf{0}}\xspace}
\newcommand{\mbl}{\ensuremath{\boldsymbol{\lambda}}\xspace}
\newcommand{\mbs}{\ensuremath{\mathbf{s}}\xspace}
\newcommand{\mbF}{\ensuremath{\mathbf{F}}\xspace}
\newcommand{\mbX}{\ensuremath{\mathbf{X}}\xspace}
\newcommand{\mbmu}{\ensuremath{\boldsymbol{\mu}}\xspace}
\newcommand{\mbmub}{\ensuremath{\mathbf{\bar{\boldsymbol{\mu}}}}\xspace}
\newcommand\inv[1]{#1\raisebox{0.7ex}{$\scriptscriptstyle-\!1$}}
\newcommand*{\plimsoll}{{\ensuremath{-\kern-4pt{\ominus}\kern-4pt-}}\xspace}
\newcommand{\formation}[1]{\ensuremath{\Delta_{\text{f}} #1^\plimsoll}\xspace}
\newcommand{\pure}{\ensuremath{\text{pure}}\xspace}
\newcommand{\lama}{\ensuremath{{\lambda_{\text{a}}}}\xspace}
\newcommand{\lamr}{\ensuremath{{\lambda_{\text{r}}}}\xspace}
\newcommand{\kB}{\ensuremath{k_{\text{B}}}\xspace}
\newcommand{\gdhs}{\ensuremath{g_{\text{d}}^\hs}\xspace}
\newcommand{\aS}{\ensuremath{a_{1}^{\text{S}}}\xspace}
\newcommand{\aSt}{\ensuremath{\tilde{a}_{1}^{\text{S}}}\xspace}
\newcommand{\Bt}{\ensuremath{\tilde{B}}\xspace}
\newcommand{\Dt}{\ensuremath{\tilde{D}}\xspace}
\newcommand{\Qt}{\ensuremath{\tilde{Q}}\xspace}
\newcommand{\at}{\ensuremath{\tilde{a}}\xspace}
\newcommand{\ab}{\ensuremath{\bar{a}}\xspace}
\newcommand{\aSij}{\ensuremath{a_{1,ij}^{\text{S}}}\xspace}
\newcommand{\aStij}{\ensuremath{\tilde{a}_{1,ij}^{\text{S}}}\xspace}
\newcommand*{\aSl}[1]{\ensuremath{a_{1{#1}}^{\text{S}}}\xspace}
\newcommand*{\aSlt}[1]{\ensuremath{\tilde{a}_{1{#1}}^{\text{S}}}\xspace}
\newcommand{\eff}{\ensuremath{\text{eff}}\xspace}
\newcommand{\gamc}{\ensuremath{\gamma_{\text{C}}}\xspace}
\newcommand{\z}{\zeta}
\newcommand{\zb}{\bar{\zeta}}
\newcommand{\nc}{\ensuremath{\text{N}}\xspace}
\newcommand{\NA}{\ensuremath{\text{N}_{\text{A}}}\xspace}
\newcommand{\KB}{\ensuremath{\text{k}_{\text{B}}}\xspace}
\newcommand{\att}{\ensuremath{\text{a}}\xspace}
\newcommand{\rep}{\ensuremath{\text{r}}\xspace}
\newcommand{\cut}{\ensuremath{\text{c}}\xspace}
\newcommand{\cs}{\ensuremath{\text{cs}}\xspace}
\newcommand{\shift}{\ensuremath{\text{s}}\xspace}
\newcommand{\reduced}{\ensuremath{\text{Red}}\xspace}
\newcommand{\Qone}{\ensuremath{\text{Q}_1}\xspace}
\newcommand{\Qtwo}{\ensuremath{\text{Q}_2}\xspace}
\newcommand{\Q}{\ensuremath{\text{Q}}\xspace}
\newcommand{\kF}{\ensuremath{\text{k}_\text{c}}\xspace}
\newcommand{\kC}{\ensuremath{\text{k}_\text{cs}}\xspace}

\title{Truncation corrections for equations of state}
\author{Morten Hammer}

\graphicspath{{gfx/}}

\begin{document}
\frontmatter
\tableofcontents
\section{Introduction}

\citet[Eq. 17]{Johnson1993} derive an equation describing the
approximate contribution form the truncated part of the interaction
potential, in their case the Lennard-Jones fluid was considered. The
derivation express the change in the Helmholtz free energy due to a
change in potential as a functional differential,

\begin{equation}
  \label{eq:dadphi}
  \frac{\delta A}{\delta \phi} = \rho^2 g \lb \vektor{r}_1,\vektor{r}_2 \rb,
\end{equation}
where $A$ is the Helmholtz free energy, $\phi$ is the potential acting
between particles, $g \lb \vektor{r}_1,\vektor{r}_2 \rb$ is the pair
correlation function, and $\vektor{r}_i$ is the position vector of a
molecule.

\citet[Eq. 17]{Johnson1993} defines the following reduced properties,

\begin{align}
  \label{eq:reduced}
  T^* &= \frac{\KB T}{\epsilon},\\
  \rho^* &= \frac{N \sigma^3}{V},\\
  A^* &= \frac{A}{N \epsilon},
\end{align}
and they will also be used in this memo.

In the following we will derive the same properties for the quantum corrected Mie potential.

\section{The potential}
The Mie potential is expressed as follows,
\begin{equation}
  \label{eq:Mie}
  \phi^\mie\lb r\rb = \mathcal{C}\epsilon \lb \lb\frac{\sigma}{r} \rb^\lamr - \lb\frac{\sigma}{r} \rb^\lama \rb,
\end{equation}
where,
\begin{equation}
  \label{eq:C}
  \mathcal{C} = \frac{\lamr}{\lamr - \lama}\lb\frac{\lamr}{\lama} \rb^{\frac{\lama}{\lamr - \lama}},
\end{equation}

The quantum correction of the Mie potential to first order, using the
Feynman Hibbs approach, take the following form,
\begin{align}
  \label{eq:Mie_Q1}
  \phi^{\Qone,\mie}\lb r, T\rb =& \mathcal{C}\epsilon D \frac{1}{r^2} \lb Q_{1}\lb \lamr \rb\lb\frac{\sigma}{r}\rb^\lamr - Q_{1}\lb \lama \rb\lb\frac{\sigma}{r}\rb^\lama\rb.
\end{align}
The second order correction becomes,
\begin{align}
  \label{eq:Mie_Q2}
  \phi^{\Qtwo,\mie}\lb r, T\rb =& \mathcal{C}\epsilon \frac{D^2}{2} \frac{1}{r^4} \lb Q_{2}\lb \lamr \rb\lb\frac{\sigma}{r}\rb^\lamr - Q_{2}\lb \lama \rb\lb\frac{\sigma}{r}\rb^\lama  \rb.
\end{align}
Here we have used the following definitions,
\begin{align}
  \label{eq:Mie_Quantum_var}
  D =& \frac{\beta \hbar^2 }{24 \mu},\\
  \beta &= \frac{1}{\KB T},\\
  h &= 2\pi \hbar,\\
  Q_{1}\lb \lambda \rb =& \lambda\lb \lambda - 1 \rb, \\
  Q_{2}\lb \lambda \rb =& \lb \lambda + 2 \rb\lb \lambda + 1\rb\lambda\lb \lambda - 1 \rb.
\end{align}
$\mu$ is molecular mass.

The overall quantum corrected Mie potential then take the following form,
\begin{align}
  \label{eq:Mie_Quantum}
  \phi\lb r, T\rb =& \phi^\mie\lb r\rb + \phi^{\Qone,\mie}\lb r, T\rb + \phi^{\Qtwo,\mie}\lb r, T\rb.
\end{align}

In the following, for simplicity, we drop writing out the temperature
dependence of the potential explicitly.

For the change in going from a cut potential, truncated at $r_\cut$, to
a full quantum corrected potential, we get,
\begin{align}
  \label{eq:dphi}
  \delta \phi_\cut\lb r\rb =& \phi\lb r\rb - \phi_\cut\lb r\rb =
  \begin{cases}
    0 & \text{if } r \leq r_\cut\\
    \phi\lb r\rb & \text{if } r > r_\cut\\
  \end{cases}.
\end{align}
Here $\phi_\cut$ is the potential cut at $r_\cut$.

For the change in going from a cut and shifted potential, truncated at $r_\cut$, to
a full quantum corrected potential, we get,
\begin{align}
  \label{eq:dphi_shift}
  \delta \phi_\cs\lb r\rb =& \phi\lb r\rb - \phi_\cs\lb r\rb =
  \begin{cases}
    \phi\lb r_\cut\rb & \text{if } r \leq r_\cut\\
    \phi\lb r\rb & \text{if } r > r_\cut\\
  \end{cases}.
\end{align}

For the change in going from a cut to a cut and shifted potential, we
get by combining equations \ref{eq:dphi} and \ref{eq:dphi_shift},
\begin{align}
  \label{eq:dphi_shift_only}
  \delta \phi_{\cut-\cs}\lb r\rb =& \phi_\cut\lb r\rb - \phi_\cs\lb r\rb =
  \begin{cases}
    \phi\lb r_\cut\rb & \text{if } r \leq r_\cut\\
    0 & \text{if } r > r_\cut\\
  \end{cases}.
\end{align}
\section{The Helmholtz free energy truncation correction}

Using equations \ref{eq:dadphi} and \ref{eq:dphi}, the change in
Helmholtz free energy due to truncation is,
\begin{align}
  \label{eq:dA}
  \Delta A_\cut = A - A_\cut &= 2 \pi N \rho \int_0^\infty g\lb r \rb \delta \phi_\cut\lb r\rb r^2 dr \\
  &= 2 \pi N \rho \int_{r_\cut}^\infty g\lb r \rb \phi\lb r\rb r^2 dr
\end{align}
Assuming, as \citeauthor{Johnson1993}, that $g\lb r \rb = 1$ for $r >
r_\cut$, the integration becomes simple. The Mie integral becomes,
\begin{align}
  \label{eq:i_mie}
  \int_{r_\cut}^\infty \phi^\mie\lb r\rb r^2 dr &= \mathcal{C}\epsilon \sigma^3
  \lb \frac{1}{\lamr - 3}\lb\frac{\sigma}{r_\cut} \rb^{\lb\lamr-3\rb} - \frac{1}{\lama - 3}\lb\frac{\sigma}{r_\cut} \rb^{\lb \lama - 3\rb} \rb\\
  &= \mathcal{C}\epsilon \sigma^3 \Lambda
\end{align}

The integral for the first order quantum correction to the Mie potential becomes,
\begin{align}
  \label{eq:i_mie_q1}
  \int_{r_\cut}^\infty \phi^{\Qone,\mie}\lb r\rb r^2 dr &= \mathcal{C}\epsilon \sigma D
                                                          \lb \frac{Q_1\lb\lamr\rb}{\lamr - 1}\lb\frac{\sigma}{r_\cut} \rb^{\lb\lamr-1\rb} - \frac{Q_1\lb\lama\rb}{\lama - 1}\lb\frac{\sigma}{r_\cut} \rb^{\lb \lama - 1\rb} \rb\\
  &= \mathcal{C}\epsilon \sigma D
  \lb \lamr\lb\frac{\sigma}{r_\cut} \rb^{\lb\lamr-1\rb} - \lama\lb\frac{\sigma}{r_\cut} \rb^{\lb \lama - 1\rb} \rb\\
  &= \mathcal{C}\epsilon \sigma D \Lambda^{\Qone}
\end{align}

The integral for the second order quantum correction to the Mie potential becomes,
\begin{align}
  \label{eq:i_mie_q2}
  \int_{r_\cut}^\infty \phi^{\Qtwo,\mie}\lb r\rb r^2 dr &= \mathcal{C}\epsilon \frac{1}{\sigma} \frac{D^2}{2}
  \lb \frac{Q_2\lb\lamr\rb}{\lamr+1}\lb\frac{\sigma}{r_\cut} \rb^{\lb\lamr+1\rb} - \frac{Q_2\lb\lama\rb}{\lama+1}\lb\frac{\sigma}{r_\cut} \rb^{\lb \lama + 1\rb} \rb\\
  &= \mathcal{C}\epsilon \frac{1}{\sigma} \frac{D^2}{2} \Lambda^{\Qtwo}
\end{align}
The reduced Helmholtz free energy change then becomes,
\begin{align}
  \label{eq:dAred}
  A^* - A_\cut^* &= 2 \pi \rho^* \mathcal{C} \left[ \Lambda + D \frac{\Lambda^{\Qone}}{\sigma^2}
    + \frac{D^2}{2} \frac{\Lambda^{\Qtwo}}{\sigma^4} \right]
\end{align}
\subsection{The Helmholtz free energy truncation correction for Thermopack}
\begin{align}
  \label{eq:dA_RT}
  F^\cut = \frac{\Delta A_\cut}{R T} = \frac{A - A_\cut}{R T} &= 2 \pi \NA \sigma^3 \mathcal{C} \left[\frac{\epsilon}{\KB}\right] \frac{n^2}{V} \frac{1}{T} \left[ \Lambda + D \frac{\Lambda^{\Qone}}{\sigma^2}
    + \frac{D^2}{2} \frac{\Lambda^{\Qtwo}}{\sigma^4} \right]
\end{align}
Introducing $\kF = 2 \pi \NA \sigma^3 \mathcal{C}
\left[\frac{\epsilon}{\KB}\right]$, the differentials of $F^\cut$ becomes,
\begin{align}
  \label{eq:dF}
  F^\cut_n =& 2 \frac{F^\cut}{n}, \\
  F^\cut_{nn} =& 2 \frac{F^\cut}{n^2}, \\
  F^\cut_{V} =& - \frac{F^\cut}{V}, \\
  F^\cut_{VV} =& 2 \frac{F^\cut}{V^2}, \\
  F^\cut_{Vn} =&  -2 \frac{F^\cut}{V n}, \\
  F^\cut_{T} =& - \frac{F^\cut}{T} + \kF \frac{n^2}{V} \frac{1}{T} \left[ D_{T} \frac{\Lambda^{\Qone}}{\sigma^2}
    + D D_{T} \frac{\Lambda^{\Qtwo}}{\sigma^4} \right], \\
  F^\cut_{Tn} =&  2 \frac{F^\cut_T}{n}, \\
  F^\cut_{TV} =&  - \frac{F^\cut_T}{V}, \\
  F^\cut_{TT} =& - 2 \frac{F^\cut}{T^2} -  2 \kF \frac{n^2}{V} \frac{1}{T^2} \left[ D_{T} \frac{\Lambda^{\Qone}}{\sigma^2}
    + D D_{T} \frac{\Lambda^{\Qtwo}}{\sigma^4} \right]  \nonumber \\ & + \kF \frac{n^2}{V} \frac{1}{T} \left[ D_{TT} \frac{\Lambda^{\Qone}}{\sigma^2} +
    \lb D_{T}^2 + D D_{TT} \rb \frac{\Lambda^{\Qtwo}}{\sigma^4} \right].
\end{align}

\section{The Helmholtz free energy shift correction}

Using equations \ref{eq:dadphi}, \ref{eq:dphi} \ref{eq:dphi_shift}, the change in
Helmholtz free energy due to truncation is,

\begin{align}
  \label{eq:dAcs}
  \Delta A_\cs &= A - A_\cut - A_\shift = \Delta A_\cut + \Delta A_{\cut-\cs}
\end{align}

\begin{align}
  \label{eq:dAscs}
  \Delta A_{\cut-\cs}  &= 2 \pi N \rho \int_0^{r_\cut} g\lb r \rb \delta \phi_\cs\lb r\rb r^2 dr \\
  &= 2 \pi N \rho \phi\lb r_\cut\rb \int_0^{r_\cut} g\lb r \rb r^2 dr
\end{align}
\citeauthor{Johnson1993} used that
$2 \pi \rho \int_0^{r_\cut} g\lb r \rb r^2 dr$ is just the number of pairs of
atoms within the cutoff of a central atom. This can be approximate by the
average number of pairs of atoms in the volume of a sphere of radius $r_\cut$.

The integration then becomes simple. And the quantum corrected Mie integral becomes,
\begin{align}
  \label{eq:i_mie_shift}
  \Delta A_{\cut-\cs}  &= \frac{2}{3} \pi N \rho \phi\lb r_\cut\rb r_\cut^3\\
                       &= \frac{2}{3} \pi N \rho \mathcal{C}\epsilon \sigma^3 \left[\Lambda_\shift
                         + D \frac{\Lambda_\shift^{\Qone}}{\sigma^2}
                         + \frac{D^2}{2} \frac{\Lambda_\shift^{\Qtwo}}{\sigma^4}\right]
\end{align}

\begin{align}
  \label{eq:lambda_shift}
  \Lambda_\shift  &= \lb\frac{\sigma}{r_\cut} \rb^{\lb\lamr-3\rb} - \lb\frac{\sigma}{r_\cut} \rb^{\lb \lama - 3\rb} \\
  \Lambda_\shift^{\Qone} &= Q_1\lb\lamr\rb\lb\frac{\sigma}{r_\cut} \rb^{\lb\lamr-1\rb} - Q_1\lb\lama\rb\lb\frac{\sigma}{r_\cut} \rb^{\lb \lama - 1\rb}\\
  \Lambda_\shift^{\Qtwo} &= Q_2\lb\lamr\rb\lb\frac{\sigma}{r_\cut} \rb^{\lb\lamr+1\rb} - Q_2\lb\lama\rb\lb\frac{\sigma}{r_\cut} \rb^{\lb \lama + 1\rb}
\end{align}

In reduced variables this becomes,
\begin{align}
  \label{eq:i_mie_shift_star}
  \Delta A_{\cut-\cs}^*  &= \frac{2}{3} \pi \rho^* \mathcal{C} \left[\Lambda_\shift
                         + D \frac{\Lambda_\shift^{\Qone}}{\sigma^2}
                         + \frac{D^2}{2} \frac{\Lambda_\shift^{\Qtwo}}{\sigma^4}\right].
\end{align}

\subsection{The Helmholtz free energy potential shift correction for Thermopack}
\begin{align}
  \label{eq:dAcs_RT}
  F^\cs = \frac{\Delta A_{\cut-\cs}}{R T} = \frac{A_\cut - A_\cs}{R T} &= \frac{2}{3} \pi \NA \sigma^3 \mathcal{C} \left[\frac{\epsilon}{\KB}\right] \frac{n^2}{V} \frac{1}{T} \left[ \Lambda_\shift + D \frac{\Lambda_\shift^{\Qone}}{\sigma^2}
    + \frac{D^2}{2} \frac{\Lambda_\shift^{\Qtwo}}{\sigma^4} \right]
\end{align}
Introducing $\kC = 2 \pi \NA \sigma^3 \mathcal{C}/3 = \kF/3
\left[\frac{\epsilon}{\KB}\right]$, the differentials of $F^\cs$ becomes,
\begin{align}
  \label{eq:dFcs}
  F^\cs_n =& 2 \frac{F^\cs}{n}, \\
  F^\cs_{nn} =& 2 \frac{F^\cs}{n^2}, \\
  F^\cs_{V} =& - \frac{F^\cs}{V}, \\
  F^\cs_{VV} =& 2 \frac{F^\cs}{V^2}, \\
  F^\cs_{Vn} =&  -2 \frac{F^\cs}{V n}, \\
  F^\cs_{T} =& - \frac{F^\cs}{T} + \kC \frac{n^2}{V} \frac{1}{T} \left[ D_{T} \frac{\Lambda_\shift^{\Qone}}{\sigma^2}
    + D D_{T} \frac{\Lambda_\shift^{\Qtwo}}{\sigma^4} \right], \\
  F^\cs_{Tn} =&  2 \frac{F^\cs_T}{n}, \\
  F^\cs_{TV} =&  - \frac{F^\cs_T}{V}, \\
  F^\cs_{TT} =& - 2 \frac{F^\cs}{T^2} -  2 \kC \frac{n^2}{V} \frac{1}{T^2} \left[ D_{T} \frac{\Lambda_\shift^{\Qone}}{\sigma^2}
    + D D_{T} \frac{\Lambda_\shift^{\Qtwo}}{\sigma^4} \right]  \nonumber \\ & + \kC \frac{n^2}{V} \frac{1}{T} \left[ D_{TT} \frac{\Lambda_\shift^{\Qone}}{\sigma^2} +
    \lb D_{T}^2 + D D_{TT} \rb \frac{\Lambda_\shift^{\Qtwo}}{\sigma^4} \right].
\end{align}
\clearpage
\bibliographystyle{plainnat}
\bibliography{../thermopack}

\end{document}
