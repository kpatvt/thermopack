\documentclass[english]{../thermomemo/thermomemo}
\usepackage[utf8]{inputenc}

\title{Hyperdual numbers}
\author{Morten Hammer}

\usepackage[normalem]{ulem}

\usepackage[numbers]{natbib}
\usepackage{hyperref}
\usepackage{color}

\definecolor{midnightblue}{RGB}{35,35,132}
\definecolor{urlblue}{RGB}{70,130,180}

\definecolor{shadecolor}{gray}{0.9}

\hypersetup{
    colorlinks=true,
    linkcolor=midnightblue,
    urlcolor=urlblue,
    citecolor=midnightblue,
    linktoc=page
}

\usepackage{amsmath}
\usepackage{framed}
\usepackage{siunitx,mhchem,todonotes}
\presetkeys%
    {todonotes}%
    {inline,backgroundcolor=orange}{}
\usepackage{xspace}
%\usepackage[arrowdel]{physics}
\usepackage{physics}

\newcommand{\pone}[3]{\frac{\partial #1}{\partial #2}\bigg|_{#3}}% partial
                                % derivative with information of
                                % constant variables
\newcommand*{\vektor}[1]{\boldsymbol{#1}}%
\newcommand{\eos}{\ensuremath{\text{eos}}\xspace}
\newcommand*{\lb}{\ensuremath{\left(}}
\newcommand*{\rb}{\ensuremath{\right)}}
\newcommand*{\lbf}{\ensuremath{\left[}}
\newcommand*{\rbf}{\ensuremath{\right]}}
\newcommand*{\vc}[1]{\vec{\mathbf{#1}}}%
\newcommand{\kB}{\ensuremath{\text{k}_{\text{B}}}\xspace}
\newcommand{\NA}{\ensuremath{N_{\text{A}}}\xspace}
\newcommand{\disp}{\ensuremath{\text{disp}}\xspace}
\newcommand{\residual}{\ensuremath{\text{res}}\xspace}
\newcommand{\ideal}{\ensuremath{\text{ig}}\xspace}
\newcommand*{\pd}[3][]{\frac{\partial^{#1}#2}{\partial{#3}^{#1}}}% partial derivative

\DeclareMathOperator*{\argmin}{arg\,min }



\usepackage[activate={true,nocompatibility},final,kerning=true,tracking=true,spacing=true,stretch=10,shrink=10]{microtype}
\microtypecontext{spacing=nonfrench}
\SetExtraKerning[unit=space]
    {encoding={*}, family={bch}, series={*}, size={footnotesize,small,normalsize}}
    {\textendash={400,400}, % en-dash, add more space around it
     "28={ ,150}, % left bracket, add space from right
     "29={150, }, % right bracket, add space from left
     \textquotedblleft={ ,150}, % left quotation mark, space from right
     \textquotedblright={150, }} % right quotation mark, space from left
\SetTracking{encoding={*}, shape=sc}{0}

\begin{document}
\frontmatter

\section{Hyperdual numbers}
This memo briefly describes the implementation of hyperdual numbers
for the purpose of differentiation. The concept of hyperdual numbers
is described by \citet{Rehner2021} and \citet{Fike2011}.

Third order hyperdual number,
\begin{equation}
  x = x_0 + x_1 \epsilon_1 + x_2 \epsilon_2 + x_3 \epsilon_3 + x_{12} \epsilon_{12} + x_{13} \epsilon_{13} + x_{23} \epsilon_{23} + x_{123} \epsilon_{123}
\end{equation}

The exact Taylor expansion of $f(x,y,z)$ using hyperdual numbers,
\begin{align}
  f \lb x + \epsilon_1, y + \epsilon_2, z + \epsilon_3  \rb &= f^0 + f_x^0 \epsilon_1 + f_y^0 \epsilon_2 + f_z^0 \epsilon_3 \nonumber \\&
  + f_{xy}^0 \epsilon_1 \epsilon_2 + f_{xz}^0 \epsilon_1 \epsilon_3 + f_{yz}^0 \epsilon_2 \epsilon_3 \nonumber \\&
  + f_{xyz}^0 \epsilon_1 \epsilon_2 \epsilon_3
\end{align}

Taylor expansion of function $f(x)$ yields
\begin{align}
  f(x) &=  f(x_0) + f^\prime (x_0) x_1 \epsilon_1 + f^\prime (x_0)x_2 \epsilon_2 + f^\prime (x_0)x_3 \epsilon_3 \nonumber \\
       &+ f^\prime (x_0) x_{12} \epsilon_{12} + f^\prime (x_0) x_{13} \epsilon_{13} + f^\prime (x_0) x_{23} \epsilon_{23} \nonumber \\
       & + f^\prime (x_0)x_{123} \epsilon_{123} \nonumber \\
       & + f^{\prime\prime} (x_0) x_1 x_2 \epsilon_1 \epsilon_2 + f^{\prime\prime} (x_0)x_1 x_3 \epsilon_1 \epsilon_3 + f^{\prime\prime} (x_0)x_2x_3 \epsilon_2  \epsilon_3 \nonumber \\
       & + f^{\prime\prime} (x_0) x_1 x_{23} \epsilon_1 \epsilon_{23} + f^{\prime\prime} (x_0)x_{12} x_3 \epsilon_{12} \epsilon_3 + f^{\prime\prime} (x_0)x_2x_{13} \epsilon_2  \epsilon_{13} \nonumber \\
       & + f^{\prime\prime\prime} (x_0) x_1 x_2 x_3 \epsilon_1 \epsilon_2 \epsilon_3 .
\end{align}

Note that the prefactors $1/2$ and $1/3$ cancels. Gathering terms,
\begin{align}
  f(x) &=  f(x_0) + f^\prime (x_0) x_1 \epsilon_1 + f^\prime (x_0)x_2 \epsilon_2 + f^\prime (x_0)x_3 \epsilon_3 \nonumber \\
       &+ \biggl(f^\prime (x_0) x_{12} + f^{\prime\prime} (x_0) x_1 x_2 \biggr) \epsilon_{1}\epsilon_{2} \nonumber \\
       & + \biggl(f^\prime (x_0) x_{13} + f^{\prime\prime} (x_0)x_1 x_3 \biggr)\epsilon_{1}\epsilon_{3} \nonumber \\
       & + \biggl(f^\prime (x_0) x_{23} + f^{\prime\prime} (x_0)x_2x_3  \biggr)\epsilon_{2} \epsilon_{3} \nonumber \\
       & + \biggl(f^\prime (x_0)x_{123}  +  f^{\prime\prime} (x_0) \biggl(x_1 x_{23} + x_{12} x_3 + x_2x_{13}\biggr) + f^{\prime\prime\prime} (x_0) x_1 x_2 x_3 \biggr) \epsilon_{1}\epsilon_{2}\epsilon_{3} .
\end{align}

Multiplication of two numbers,
\begin{align}
  xy &=  x_0y_0 + \lb x_0 y_1 + x_1 y_0 \rb \epsilon_1 + \lb x_0 y_2 + x_2 y_0 \rb \epsilon_2 + \lb x_0 y_3 + x_3 y_0 \rb \epsilon_3 \nonumber \\
     & + \lb x_{0} y_{12} + x_{12} y_{0} + x_{1} y_{2} + x_{2} y_{1} \rb \epsilon_1 \epsilon_2  \nonumber \\
     & + \lb x_{0} y_{13} + x_{13} y_{0} + x_{1} y_{3} + x_{3} y_{1} \rb \epsilon_1 \epsilon_3  \nonumber \\
     & + \lb x_{0} y_{23} + x_{23} y_{0} + x_{3} y_{2} + x_{2} y_{3} \rb \epsilon_2 \epsilon_3  \nonumber \\
     & + \lb x_{0} y_{123} + x_{123} y_{0} + x_{12} y_{3} + x_{3} y_{12} + x_{13} y_{2} + x_{2} y_{13} + x_{23} y_{1} + x_{23} y_{1} \rb \epsilon_1 \epsilon_2 \epsilon_3 .
\end{align}

\subsection{Needed differenitials}
Differentials to third order is required for the most common functions.

\subsubsection{Exponential function ($\exp(x)$)}
\begin{equation}
  f(x) = \exp(x) = f^\prime(x) = f^{\prime\prime}(x) = f^{\prime\prime\prime}(x)
\end{equation}
\subsubsection{Sine function ($\sin(x)$)}
\begin{align}
  f(x) &= \sin(x) \\
  f^\prime(x) &= \cos(x) \\
  f^{\prime\prime}(x) &= -\sin(x) \\
  f^{\prime\prime\prime}(x) &= -\cos(x)
\end{align}
\subsubsection{Cosine function ($\cos(x)$)}
\begin{align}
  f(x) &= \cos(x) \\
  f^\prime(x) &= -\sin(x) \\
  f^{\prime\prime}(x) &= -\cos(x) \\
  f^{\prime\prime\prime}(x) &= \sin(x)
\end{align}
\subsubsection{Tangent function ($\tan(x)$)}
\begin{align}
  f(x) &= \tan(x) \\
  f^\prime(x) &= \sec^2 (x) = \tan^2(x) + 1 \\
  f^{\prime\prime}(x) &= 2 \tan(x) \sec^2(x)  \\
  f^{\prime\prime\prime}(x) &= 2 \sec^2(x) (\sec^2(x) + 2 \tan^2(x))
\end{align}
\subsubsection{Natural logarithm ($\log(x)$)}
\begin{align}
  f(x) &= \log(x) \\
  f^\prime(x) &= \frac{1}{x} \\
  f^{\prime\prime}(x) &= -\frac{1}{x^2}\\
  f^{\prime\prime\prime}(x) &= \frac{2}{x^3}
\end{align}
\subsubsection{Inverse cosine function ($\acos(x)$)}
\begin{align}
  f(x) &= \acos(x) \\
  f^\prime(x) &= -\frac{1}{\sqrt{1 + x^2}} \\
  f^{\prime\prime}(x) &= -\frac{x}{(1 - x^2)^{3/2}}\\
  f^{\prime\prime\prime}(x) &= \frac{-2 x^2 - 1}{(1 - x^2)^{5/2}}
\end{align}
\subsubsection{Inverse sine function ($\asin(x)$)}
\begin{align}
  f(x) &= \asin(x) \\
  f^\prime(x) &= \frac{1}{\sqrt{1 + x^2}} \\
  f^{\prime\prime}(x) &= \frac{x}{(1 - x^2)^{3/2}}\\
  f^{\prime\prime\prime}(x) &= \frac{2 x^2 + 1}{(1 - x^2)^{5/2}}
\end{align}
\subsubsection{Inverse tangent function ($\atan(x)$)}
\begin{align}
  f(x) &= \atan(x) \\
  f^\prime(x) &= \frac{1}{1 + x^2} \\
  f^{\prime\prime}(x) &= \frac{-2 x}{(x^2 + 1)^2}\\
  f^{\prime\prime\prime}(x) &= \frac{6 x^2 - 2}{(x^2 + 1)^3}
\end{align}
\subsubsection{Power function ($x^a$)}
\begin{align}
  f(x) &= x^a \\
  f^\prime(x) &= a x^{a-1} \\
  f^{\prime\prime}(x) &= a (a-1)  x^{a-2} \\
  f^{\prime\prime\prime}(x) &= a (a-1) (a-2)  x^{a-3}
\end{align}

\clearpage
\bibliographystyle{plainnat}
\bibliography{../thermopack}

\end{document}

%%% Local Variables:
%%% mode: latex
%%% TeX-master: t
%%% End:
