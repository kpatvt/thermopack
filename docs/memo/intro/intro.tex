\documentclass[english]{../thermomemo/thermomemo}
% NOTE: You must pass either norsk or english as an option!
\usepackage[utf8]{inputenc}

\title{Introduction to thermodynamics}
\author{Morten Hammer, Geir Skaugen and Eskil Aursand}

\usepackage[normalem]{ulem}

\usepackage{hyperref}
\usepackage{color}
\usepackage{amsfonts}

\definecolor{midnightblue}{RGB}{35,35,132}
\definecolor{urlblue}{RGB}{70,130,180}

\definecolor{shadecolor}{gray}{0.9}

\hypersetup{
    colorlinks=true,
    linkcolor=midnightblue,
    urlcolor=urlblue,
    citecolor=midnightblue,
    linktoc=page
}

\usepackage{amsmath}
\usepackage{hyperref}
\usepackage{framed}
\usepackage{siunitx,mhchem,todonotes}
\newcommand{\pone}[3]{\frac{\partial #1}{\partial #2}\bigg|_{#3}}% partial
                                % derivative with information of
                                % constant variables
\newcommand*{\vektor}[1]{\boldsymbol{#1}}%
\newcommand{\dd}[1]{\mathrm{d}{#1}}

\DeclareMathOperator*{\argmin}{arg\,min }



\usepackage[activate={true,nocompatibility},final,kerning=true,tracking=true,spacing=true,stretch=10,shrink=10]{microtype}
\microtypecontext{spacing=nonfrench}
\SetExtraKerning[unit=space]
    {encoding={*}, family={bch}, series={*}, size={footnotesize,small,normalsize}}
    {\textendash={400,400}, % en-dash, add more space around it
     "28={ ,150}, % left bracket, add space from right
     "29={150, }, % right bracket, add space from left
     \textquotedblleft={ ,150}, % left quotation mark, space from right
     \textquotedblright={150, }} % right quotation mark, space from left
\SetTracking{encoding={*}, shape=sc}{0}

\begin{document}
\frontmatter

\tableofcontents

\section{Introduction}
This document is ment as an introduction to thermodynamics for users
and programmers in thermopack.

In addition to this memo, there exsist two memos written to document
thermopack. These can be generated running pdflatex in the thermopack
documentation folder ./doc/memo/.

Otherwise the thermodynamics book by Callen \cite{callen85} and by
Michelsen \cite{michelsen07} is recommended. Much of the work done on
flash algorithms are based on the work by Michelsen
\cite{michelsen82a,michelsen82b,michelsen99}.

\section{Thermodynamic potentials}
\subsection{Internal energy}
As a starting point, one assumes a fundamental equation for the internal energy ($U$), with the extensive variables entropy ($S$), volume ($V$) 
and molar numbers ($\vektor{n})$ as independent variables:
\begin{equation}
  U = U(S,V,\vektor{n})
  \label{}
\end{equation}
Under the assumption that $U$ is a \textit{first order homogeneous function} of the extensive variables, one may use 
\textit{Euler's theroem of homogeneous functions} to show that:
\begin{align}
  U(S,V,\vektor{n}) &= S \pone{U}{S}{V,\vektor{n}}
  +V \pone{U}{V}{S,\vektor{n}}
  +\sum_i n_i \pone{U}{n_i}{S,V,n_j} \nonumber \\
  &= ST - VP + \sum_i n_i \mu_i
  \label{eq:U_euler}
\end{align}
using the definitions 
\begin{equation}
  T \equiv \pone{U}{S}{V,\vektor{n}}, \quad P \equiv \pone{U}{V}{S,\vektor{n}}, \quad \mu_i \equiv \pone{U}{n_i}{S,V,n_j}.
  \label{}
\end{equation}
The differential of $U$ given its homogeneous property \eqref{eq:U_euler} is 
\begin{equation}
  \dd{U} = S\dd{T} + T\dd{S} - V\dd{P} - P\dd{V} + \sum_i n_i \dd{\mu_i} + \sum_i \mu_i \dd{n_i},
  \label{}
\end{equation}
while generally the total (exact) differential of \eqref{eq:U_euler}, given its independent variables, is
\begin{equation}
  \dd{U} = T\dd{S} - P\dd{V} + \sum_i \mu_i \dd{n_i},
  \label{eq:U_totaldiff}
\end{equation}
which combine into the \textit{Gibbs-Duhem} relation for the intensive properties:
\begin{equation}
  S\dd{T} - V\dd{P} + \sum_i n_i \dd{\mu_i} = 0.
  \label{eq:gibbsduhem}
\end{equation}

\subsection{Helmholtz energy}
The Helmholtz energy is defined as
\begin{equation}
  A(T,V,\vektor{n}) \equiv U - TS,
  \label{eq:helmholtz_def}
\end{equation}
and by using \eqref{eq:U_totaldiff}, one may show that
\begin{align}
  \dd{A} = -S\dd{T} - P\dd{V} + \sum_i \mu_i \dd{n_i},
  \label{}
\end{align}
and one may then see that
\begin{equation}
  S = -\pone{A}{T}{V,\vektor{n}}, \quad P = -\pone{A}{V}{T,\vektor{n}}, \quad \mu_i = \pone{A}{n_i}{T,V,n_j}.
  \label{eq:A_differentials}
\end{equation}


\subsection{Gibbs energy}
The Gibbs energy is defined as 
\begin{equation}
  G(T,P,\vektor{n}) \equiv U + PV - TS,
  \label{eq:gibbs_def}
\end{equation}
and by using \eqref{eq:U_totaldiff}, one may show that
\begin{align}
  \dd{G} = -S\dd{T} + V\dd{P} + \sum_i \mu_i \dd{n_i}.
  \label{eq:dG}
\end{align}
and one may then see that
\begin{equation}
  S = -\pone{G}{T}{P,\vektor{n}}, \quad V = \pone{G}{P}{T,\vektor{n}}, \quad \mu_i = \pone{G}{n_i}{T,P,n_j}.
  \label{}
\end{equation}

Combining \eqref{eq:gibbsduhem} with \eqref{eq:dG} leads to  
\begin{align}
  \dd{G} = \sum_i \mu_i \dd{n_i} + \sum_i n_i \dd{\mu_i},
  \label{}
\end{align}
which can be recognized as a total differential for the alternative expression for G:
\begin{equation}
  G = \sum_i \mu_i n_i.
  \label{eq:G_alt}
\end{equation}


\subsection{Enthalpy}
The Enthalpy is defined as 
\begin{equation}
  H(S,P,\vektor{n}) \equiv U + PV,
  \label{}
\end{equation}
and by using \eqref{eq:U_totaldiff}, one may show that
\begin{align}
  \dd{H} = T\dd{S} + V\dd{P} + \sum_i \mu_i \dd{n_i}.
  \label{eq:dH}
\end{align}
and one may then see that
\begin{equation}
  T = \pone{H}{S}{P,\vektor{n}}, \quad V = \pone{H}{P}{S,\vektor{n}}, \quad \mu_i = \pone{H}{n_i}{S,P,n_j}.
  \label{}
\end{equation}

\section{Equilibrium in multi-phase systems with given (T,P)}
\subsection{Systems with given (T,P)}
\label{sec:eq_givenTP}
To characterize the equilibrium state, let us image that a small perturbation from equilibrium happens, and then focus on 
the spontaneous process ($\delta$) leading back to the equilibrium state. 
For an isolated system, the second law of thermodynamics states that all spontaneous changes towards the equilibrium 
state involve an increase in entropy:
\begin{equation}
  \delta S_\text{tot} \geq 0
  \label{eq:secondlaw}
\end{equation}
where $\delta$ indicates that it is not a change between different equilibrium states, as $\mathrm{d}$ is used to indicate. 
We are however not interested in an isolated system, but rather a system with given temperature and pressure. This may be though of as a sub-system of a total 
system, the latter satisfying \eqref{eq:secondlaw}. The other part of the total system is the \textit{reservoir}, which holds the given temperature and 
pressure, and is large enough to force our system to keep the same values. Energy and volume (but not particles) 
may be freely exchanged between the system (no subscript) and 
the reservour (subscript r), in order to find the equilibrium of the total system. As entropy is an extensive quantity, the total entropy (which must increase) is 
a sum of the entropy of the system and en entropy of the reservoir:
\begin{align}
  \delta S  + \delta S_\text{r} = \delta S_\text{tot} \geq 0.
  \label{eq:secondlaw_composite}
\end{align}
Defining $\delta Q$ as the heat transferred into the system, the change of entropy in the reservoir is given by
\begin{align}
  \delta S_\text{r} = - \frac{\delta Q}{T} = - \left( \frac{1}{T}\delta U + \frac{P}{T} \delta V \right),
  \label{}
\end{align}
by using the first and second laws of thermodynamics. The implication of \eqref{eq:secondlaw_composite} is then that
\begin{align}
  T \delta S \geq \delta U +  P \delta V.
  \label{eq:TdS}
\end{align}
The change in Gibbs energy at constant temperature and pressure is 
\begin{align}
  \delta G &= \delta \left( U + PV - TS \right) \nonumber\\ 
  &= \delta U + P \delta V + V \delta P - T \delta S - S \delta T \nonumber\\
  &= \delta U + P \delta V - T \delta S,
  \label{}
\end{align}
which combined with \eqref{eq:TdS} shows that
\begin{equation}
  \delta G \leq 0.
  \label{}
\end{equation}
In words, this shows that a process which increases the entropy of the total system must decrease the Gibbs energy of a sub-system, given that the rest of the 
total system behaves like a $T,P$ reservoir. Given that this process characterizes returns from perturbations around the equilibrium state, the 
conclusion must be:
\begin{framed}
  \noindent
  For a closed system of given temperature and pressure, the equilibrium condition is that its Gibbs energy ($G$) is at a minimum. 
\end{framed}

\subsection{Multi-phase systems}
\label{sec:eq_multiphase}
So far the internal workings of the system have been ignored. Now we want to take into account that the system may consist of multiple phases, 
which essentially are sub-systems which are free to exchange energy, volume and particles.
As shown in Sec.~\ref{sec:eq_givenTP}, the condition for equilibrium is that the Gibbs energy for the entire system is at an extremum with respect to perturbation. 
Let us assume that the perturbation is sufficiently small that the chemical potentials to not change, \textit{i.e.} $\delta \mu_i \approx 0$. This may also be seen as 
assuming that the perturbation is so small that $\delta G \approx \dd{G}$, where the latter is given by \eqref{eq:dG}.
For constant $T$ and $P$ this means that 
\begin{align}
  \delta G = \sum_j \delta G_j = \sum_j \sum_i \mu_i^{(j)} \delta n_i^{(j)} = 0,
  \label{eq:dG_multiphase}
\end{align}
where $j$ indicates the phase, and $i$ indicates the component. The conservation of matter, given no chemical reactions, 
gives that any change in the mole number of a component in an arbitrary phase $\alpha$, must come from a net opposite change in the same 
component in the other phases:
\begin{equation}
  \delta n_i^{(\alpha)}  = - \sum_{j\neq \alpha} \delta n_i^{(j)}.
  \label{eq:matter_cons}
\end{equation}
Combining \eqref{eq:dG_multiphase} and \eqref{eq:matter_cons} leads to 
\begin{equation}
  \sum_{j\neq \alpha} \sum_i \left( \mu_i^{(j)} - \mu_i^{(\alpha)} \right) \delta n_i^{(j)} = 0.
  \label{}
\end{equation}
Given that the above must be true for any variation of the independent mole numbers, this means that 
\begin{equation}
  \mu_i^{(\alpha)} = \mu_i^{(j)} \quad \forall \quad  i,j
  \label{}
\end{equation}
which in words is stated as:
\begin{framed}
  \noindent
  For a multi-phase system of given temperature and pressure, the equilibrium condition is that for each component, the chemical potential 
  is equal in all phases.
\end{framed}

Additionally, there is a stability criterion for each phase, stated as:
\begin{framed}
  \noindent
  For a multi-phase system of given temperature and pressure, the condition for phase stability is that the change in total Gibbs energy associated with 
  the formation of a new infinitesimal phase within an existing phase is non-negative, for \textit{any} composition of the new phase. 
  This is called the \textit{tangent plane condition}.
\end{framed}
The change in Gibbs energy with the formation of a new phase of $\delta n$ moles and composition $w_i$, from an existing phase of composition $z_i$, is
\begin{align}
  \delta G 
  &= \sum_i \mu_i(\vektor{z})\left( - w_i \delta n \right) + \sum_i \mu_i(\vektor{w})\left( + w_i \delta n \right) \nonumber\\
  &= \delta n \sum_i w_i \left( \mu_i(\vektor{w}) - \mu_i(\vektor{z}) \right), 
  \label{}
\end{align}
which must be non-negative for any composition $\vektor{w}$, for the existing phase composition $\vektor{z}$:
\begin{align}
  \sum_i w_i \left( \mu_i(\vektor{w}) - \mu_i(\vektor{z}) \right) \geq 0.
  \label{eq:tangentplane}
\end{align}
When checking the stability of equilibrium phases, the chemical potential of each component is equal between the phases, and thus the left hand side of 
\eqref{eq:tangentplane} is identical for each phase.



\section{Equilibrium in terms of an equation of state}
\subsection{The equation of state}
An \textit{equation of state} is often given as an explicit pressure equation, such as 
\begin{equation}
  P = P (T,V,\vektor{n}).
  \label{}
\end{equation}
However, many equations of state is expressed as a Helmholtz energy
function, 
\begin{equation}
  A = A(T,V,\vektor{n}).
  \label{}
\end{equation}
which is useful because all thermodynamical properties
may be derived by differentiating it.

The Helmholtz function of new multi--parameter GERG equations of state
\cite{kunz07,span96} and the older cubic equations of state is represented as a sum of 
an ideal contribution, and a departure (residual) contribution:
\begin{equation}
\label{eq:helmholtz}
A(T,V,\vektor{n}) = A^{\text{ig}}(T,V,\vektor{n}) + A^{\text{r}}(T,V,\vektor{n})
\end{equation}

Often a reduced (dimensionless) form of the residual Helmholtz energy is used,
\begin{equation}
\label{eq:reduced_helmholtz}
F(T,V,\vektor{n}) = a \equiv  \frac{A^\text{r}(T,V,\vektor{n})}{nRT}
\end{equation}

\subsection{The residual Helmholtz energy}

The relationship between pressure and the Helmholtz energy two forms is:
\begin{equation}
  P(T,V,\vektor{n}) = -\pone{A}{V}{T,\vektor{n}} 
  = \frac{nRT}{V} - \pone{A^{\text{r}}(T,V,\vektor{n})}{V}{T,\vektor{n}}.
  \label{}
\end{equation}
An expression for $A^\text{r}$ may be found by integrating to the limit of infinite volume at constant temperature and mole numbers:
\begin{equation}
  A^\text{r}(T,V,\vektor{n}) 
  = A^\text{r}(T,\infty,\vektor{n}) + \int_\infty^V \left[ P^\text{ig}(T,V^\prime,\vektor{n}) - P(T,V^\prime,\vektor{n}) \right]\dd{V^\prime},
  \label{}
\end{equation}
which when using that real fluids behave as an ideal gas in the limit of zero pressure ($V \to \infty$) leads to the expression:
\begin{framed}
\begin{align}
  A^\text{r}(T,V,\vektor{n}) 
  = \int^\infty_V \left[ P(T,V^\prime,\vektor{n}) - \frac{nRT}{V^\prime} \right]\dd{V^\prime}
  \label{eq:helmholtz_int}
\end{align}
\end{framed}

Be aware that residual quantities will depend on which set of variables they are represented with. As an example, let's show 
how this works out for the residual Helmholtz energy:
From the state $(T,V,P,\vektor{n})$ by definition being a real state, we know that 
\begin{equation}
  A(T,V,\vektor{n}) = A(T,P,\vektor{n}).
  \label{eq:Aequal}
\end{equation}
While the value of $A$ is independent of representation, the distribution of it between the ideal 
contribution and the residual is not. This is so because  
the state $(T,V,P,\vektor{n})$ is not in 
general a valid state of an ideal gas, so 
generally we have that
\begin{equation}
  A^\text{ig}(T,V,\vektor{n}) \neq A^\text{ig}(T,P,\vektor{n}).
  \label{}
\end{equation}

Using an arbitrary valid ideal gas state at temperature $T$, $(T,V_0,P_0,\vektor{n})$, we 
have by definition of this state that 
\begin{equation}
  A^\text{ig}(T,P_0,\vektor{n}) = A^\text{ig}(T,V_0,\vektor{n})
  \label{eq:Aequal_ideal}
\end{equation}

Combining \eqref{eq:helmholtz}, \eqref{eq:Aequal}, \eqref{eq:Aequal_ideal} and the ideal gas law  
leads to the following calculation 
for the difference in residuals between variable representations:
\begin{align}
  A^\text{r}(T,V,\vektor{n}) - A^\text{r}(T,P,\vektor{n}) 
  &= A^\text{ig}(T,P,\vektor{n}) - A^\text{ig}(T,V,\vektor{n})
  \nonumber \\
  &=\left[A^\text{ig}(T,P,\vektor{n}) - A^\text{ig}(T,P_0,\vektor{n})\right]
  -
  \left[A^\text{ig}(T,V,\vektor{n}) - A^\text{ig}(T,V_0,\vektor{n})\right]
  \nonumber\\
  &=\int_{P_0}^P 
  \pone{A^\text{ig}}{P}{T,\vektor{n}}
  \dd{P}
  -
  \int_{V_0}^V 
  \pone{A^\text{ig}}{V}{T,\vektor{n}}
  \dd{V}
  \nonumber\\
  &= nRT \ln\left( \frac{PV}{P_0 V_0} \right) 
  = nRT \ln\left( \frac{PV}{nRT} \right) 
  = nRT \ln\left( Z \right),
  \label{}
\end{align}
where $Z$ is the compressibility factor, defined by
\begin{equation}
  Z \equiv \frac{PV}{nRT} = 1 - \frac{V}{nRT} \pone{A^\text{r}(T,V,\vektor{n})}{V}{T,\vektor{n}} .
  \label{eq:Z_def}
\end{equation}

Thus we have the following expression for the residual Helmholtz free energy expressed through the 
variables $(T,P,\vektor{n})$:
\begin{equation}
\label{eq:helmholtz_int_TPN}
A^{\text{r}}(T,P,\vektor{n})  = \int_V^\infty \left(P(T,V^\prime,\vektor{n}) - \frac{nRT}{V^\prime}\right) \dd{V^\prime} 
-nRT \ln\left( Z \right)
\end{equation}




\subsection{Fugacity}
For a pure ideal gas, the change in chemical potential corresponding to a change in pressure, at constant temperature, is give by 
\begin{equation}
  \frac{P}{P_0} = \exp\left( \frac{\mu^\text{ig}(T,P) - \mu^\text{ig}(T,P_0) }{RT} \right).
  \label{}
\end{equation}
Real fluids do not obey this relation, but one may define an ``effective pressure'' called 
\textit{fugacity} ($f$) such that the following relation holds:
\begin{equation}
  \frac{f(T,P)}{P_0} = \exp\left( \frac{\mu(T,P) - \mu^\text{ig}(T,P_0) }{RT} \right).
  \label{}
\end{equation}
The reference pressure is arbitrary, so one may set $P_0=P$ to make an expression only in terms of the deviation from ideal gas:
\begin{equation}
  \frac{f(T,P)}{P} = \exp\left( \frac{\mu(T,P) - \mu^\text{ig}(T,P) }{RT} \right),
  \label{}
\end{equation}
which shows that for ideal gases, $f(T,P)=P$.



\subsubsection{Ideal gas mixtures}
An ideal gas mixture is defined as a mixture which has the following expression for the Helmholtz free energy:
\begin{equation}
  A^\text{ig}(T,V,\vektor{n}) \equiv \sum_i n_i \left( 
  \mu_i^\text{ig}(T,P_0) + RT\ln \frac{n_i RT}{P_0 V} - RT
  \right)
  \label{eq:def_idealmix}
\end{equation}
where $n_i$ is the mole number of component $i$, and
$\mu^\text{ig}_i(T,P)$ is the chemical potential of component $i$ in its pure form 
at the given temperature and pressure. This definition is consistent with the ideal gas law for the mixture as a whole, 
$PV=nRT$, where $n=\sum n_i$.
From \eqref{eq:def_idealmix} one may find the Gibbs free energy of the 
ideal gas mixture as
\begin{align}
  G^\text{ig}(T,P,\vektor{n}) 
  &= A^\text{ig}(T,V,\vektor{n}) + \left( PV \right)^\text{ig} = A^\text{ig}(T,V,\vektor{n}) + nRT \nonumber\\
  &= \sum_i n_i \left( 
  \mu_i^\text{ig}(T,P_0) + RT\ln \frac{x_i P}{P_0}
  \right)
  \label{eq:G_idealmix}
\end{align}
where $x_i \equiv n_i/n$ is the \textit{mole fraction} of component $i$.

From \eqref{eq:def_idealmix} and \eqref{eq:G_idealmix} 
one may find the chemical potential, in terms of $V$ or $P$, of component $i$ in the ideal mixture as
\begin{align}
  \mu_i^\text{ig}(T,V,\vektor{n}) &\equiv \pone{A^\text{ig}}{n_i}{T,V} = \mu_i^\text{ig}(T,P_0) + RT\ln \frac{n_i RT}{P_0 V} \nonumber \\
  \mu_i^\text{ig}(T,P,\vektor{n}) &\equiv \pone{G^\text{ig}}{n_i}{T,P} = \mu_i^\text{ig}(T,P_0) + RT\ln \frac{x_i P}{P_0},
  \label{eq:mu_ig}
\end{align}
which are equal if the state $(T,V,P,\vektor{n})$ is a valid ideal gas state ($PV=nRT$), as expected. Combining the two 
equations in \eqref{eq:mu_ig}, while setting $P_0=P$, leads the the following useful expression:
\begin{equation}
  \mu_i^\text{ig}(T,V,\vektor{n}) - \mu_i^\text{ig}(T,P,\vektor{n})
  =-RT\ln \left( Z \right)
  \label{eq:mu_ig_PVdiff}
\end{equation}
where if $(T,V,P,\vektor{n})$ is a valid state for an ideal gas, 
$Z=1$, and the right hand side reduces to zero as expected.

The last equation in \eqref{eq:mu_ig} may be rearranged to form the inspiration for the definition of 
real mixture fugacity in the next section:
\begin{equation}
  \frac{P_i}{P_0} = \exp\left( \frac{\mu^\text{ig}_i(T,P,\vektor{n}) - \mu^\text{ig}_i(T,P_0)}{RT} \right)
  \label{eq:p_mu_ideal}
\end{equation}
where $P_i=x_iP$ is the \textit{partial pressure} of component $i$. For $P_0 = P$ this reduces to 
\begin{equation}
  \mu^\text{ig}_i(T,P,\vektor{n}) - \mu^\text{ig}_i(T,P) = RT\ln x_i
  \label{eq:mu_and_xi_idealmix}
\end{equation}

\subsubsection{Real mixtures}
\label{sec:realmixtures}
Similar to the case of pure components, one may define an ``effective partial pressure'' of a component in a mixture, the 
component fugacity ($f_i$), such that the following relation similar to \eqref{eq:p_mu_ideal} holds:
\begin{equation}
  \frac{f_i(T,P,\vektor{n})}{P_0} \equiv \exp\left( \frac{\mu_i(T,P,\vektor{n}) - \mu_i^\text{ig}(T,P_0)}{RT} \right),
  \label{eq:fug_def_mix}
\end{equation}
where one may again set $P_0=P$ to state it in terms of deviations from a pure ideal gas:
\begin{equation}
  \frac{f_i(T,P,\vektor{n})}{P} \equiv \exp\left( \frac{\mu_i(T,P,\vektor{n}) - \mu_i^\text{ig}(T,P)}{RT} \right),
  \label{eq:fug_def_mix_P0eqP}
\end{equation}

The chemical potential of a component in a real mixture may be decomposed into an ideal gas part and a residual part:
\begin{align}
  \mu_i(T,V,\vektor{n}) &= \mu_i^\text{ig}(T,V,\vektor{n}) + \mu_i^\text{r}(T,V,\vektor{n}) \nonumber\\
  \mu_i(T,P,\vektor{n}) &= \mu_i^\text{ig}(T,P,\vektor{n}) + \mu_i^\text{r}(T,P,\vektor{n})
  \label{}
\end{align}
which may be combined with \eqref{eq:mu_ig_PVdiff} to show that 
\begin{equation}
  \mu_i^\text{r}(T,V,\vektor{n}) - \mu_i^\text{r}(T,P,\vektor{n})
  =RT\ln \left( Z \right)
  \label{eq:mu_r_PVdiff}
\end{equation}

Combining \eqref{eq:fug_def_mix_P0eqP} and \eqref{eq:mu_and_xi_idealmix}, one finds that
\begin{equation}
  \frac{f_i(T,P,\vektor{n})}{x_i P} \equiv
  \exp\left( \frac{\mu_i(T,P,\vektor{n}) - \mu_i^\text{ig}(T,P,\vektor{n})}{RT} 
  \right)
  \label{eq:fugcoeff_vs_mu}
\end{equation}

The \textit{fugacity coefficient} may now be defined, calculated using \eqref{eq:fugcoeff_vs_mu} 
and \eqref{eq:mu_r_PVdiff} as follows:
\begin{align}
  \phi_i(T,P,\vektor{n}) \equiv \frac{f_i(T,P,\vektor{n})}{x_i P} &= 
  \exp\left( \frac{\mu_i(T,P,\vektor{n}) - \mu_i^\text{ig}(T,P,\vektor{n})}{RT}\right)
  \nonumber\\
  &= \exp\left( \frac{\mu_i^\text{r}(T,P,\vektor{n})}{RT} \right)
  \nonumber\\
  &= \exp\left( \frac{\mu_i^\text{r}(T,V,\vektor{n})}{RT} - \ln\left( Z \right) \right)
  \nonumber\\
  &= \exp\left( \frac{1}{RT}\pone{A^\text{r}(T,V,\vektor{n})}{n_i}{T,V,n_j} - \ln\left( Z \right) \right)
  \label{eq:fugcoeff}
\end{align}
where $A^\text{r}(T,V,\vektor{n})$ may be found from \eqref{eq:helmholtz_int}.

As shown in Sec.~\ref{sec:eq_multiphase}, the condition of phase equilibrium is that for each component, 
the chemical potential must be equal in all phases. Using \eqref{eq:fug_def_mix_P0eqP}, and the fact that 
$\mu_i^\text{ig}(T,P)$ is independent of phase, the condition may be restated in the more useful form:
\begin{framed}
  \noindent
  For a multi-phase system of given temperature and pressure, the equilibrium condition is that for each component $i$, the quantity 
  $x_i \phi_i$ is equal in all phases. The fugacity coefficient may be calculated from the equation of state through 
  \eqref{eq:fugcoeff} and \eqref{eq:Z_def}.
\end{framed}

\section{TP Flash}

\subsection{General equations of phase equilibrium and stability}
As shown in Sec.~\ref{sec:eq_multiphase} and~\ref{sec:realmixtures}, the criterion for equilibrium in a multi-phase system at given temperature 
and pressure is that for each component, the fugacity is equal in all phases.
\begin{equation}
  f_i^{(\alpha)} = f_i^{(j)} \quad \forall \quad i,j,\alpha
  \label{}
\end{equation}

Additionally, there is a phase stability criterion, stated as the \textit{tangent plane condition} in Sec.~\ref{sec:eq_multiphase}.
Let us define a \textit{tangent plane distance function} for the composition $\vektor{z}$, a function of a trial composition $\vektor{w}$:
\begin{equation}
  \mathit{TPD}_z(\vektor{w}) \equiv \sum_{i} w_i \left( \mu_i(\vektor{w}) - \mu_i(\vektor{z}) \right).
  \label{}
\end{equation}
It is often conventient to use a 
\textit{reduced tangent plane distance function}
\begin{align}
  \mathit{tpd}_z(\vektor{w}) &\equiv \frac{\mathit{TPD}(\vektor{w})}{RT} 
  \nonumber\\
  &= \sum_{i} w_i \left[ \ln w_i + \ln \phi_i(\vektor{w}) - \ln z_i - \ln \phi_i(\vektor{z}) \right]
  \nonumber\\
  &= \sum_{i} w_i \left[ \ln w_i + \ln \phi_i(\vektor{w}) - d_i(\vektor{z}) \right],
  \label{}
\end{align}
where $d_i \equiv \ln z_i + \ln \phi_i(\vektor{z}) $

The phase stability criterion is then
\begin{equation}
  \mathit{tpd}_z(\vektor{w}) \geq 0 \quad \text{for any valid composition $\vektor{w}$}
  \label{}
\end{equation}
and this may be shown to be both a necessary and sufficient condition for phase stability \cite{michelsen07}. 
In practice, ensuring that $\mathit{tpd}_z(\vektor{w})$ is non-negative everywhere in compositional space is done by finding its 
minima, and then checking if it is negative at these locations. Finding the minima of $\mathit{tpd}_z(\vektor{w})$, subject to the constraint of 
$\sum w_i = 1$, may be solved using the \textit{method of Lagrange multipliers}.

Modified (unconstrained) formulation:
\begin{align}
  \mathit{tm}_z(\vektor{W}) &= 1 + \sum_{i} W_i \left[ \ln W_i + \ln \phi_i(\vektor{W}) - d_i(\vektor{z}) -1\right]
  \nonumber\\
  &= \left( 1 - W_T + W_T \ln W_T \right) + W_T \mathit{tpd}_z(\vektor{w})
  \label{eq:tm}
\end{align}
where $W_i = w_i W_T$ ($W_T \equiv \sum W_i $) are mole numbers, not composition, and are thus not constrained to sum to one.
At stationary points of $\mathit{tm}_z(\vektor{W})$ with respect to $W_i$:
\begin{equation}
  \ln W_i^\text{SP} + \ln \phi_i(\vektor{W}^\text{SP}) - d_i(\vektor{z}) = 0 \quad \forall \quad i
  \label{eq:tm_stationary}
\end{equation}
\begin{equation}
  \mathit{tm}_z(\vektor{W}^\text{SP}) = 1 - W_T^\text{SP} 
  \label{}
\end{equation}

It may be shown~\cite[Ch.~9]{michelsen07} that the stationary points of
$\mathit{tm}_z(\vektor{W})$ are also valid stationary points for the
constrained function $\mathit{tpd}_z(\vektor{w})$, and
$\mathit{tpd}_z(\vektor{w}^\text{SP})$ is positive at these minima if and only
if $W_T^\text{SP} \leq 1$, \textit{i.e.} if
$\mathit{tm}_z(\vektor{W}^\text{SP}) \geq 0$. The procedure to check stability
can then be formulated as:
\begin{framed}
\begin{enumerate}
  \item Find the stationary points of $\mathit{tm}_z(\vektor{W})$ with respect to $\vektor{W}$, \textit{i.e} solutions of \eqref{eq:tm_stationary}.
  \item Check if $W_T^\text{SP} \leq 1$ at all the stationary points found. If not, the mixture $\vektor{z}$ is unstable.
\end{enumerate}
\end{framed}
Eq.~\eqref{eq:tm_stationary} may be solved by \textit{successive substitution}
or \textit{Newton's method}. As discussed by Michelsen \cite{michelsen07},
successive substitution will converge in most case but may suffer from a low
rate of convergence, giving merit to the use of Newton's method. 

A composition $\vektor{z}$ is metastable as long as the Hessian matrix 
$H(\vektor{z})$ 
of $\mathit{tm}_z(\vektor{W})$ is positive definite. 
If not, the state is \textit{intrinsically unstable}. 
The limit between the two cases is the 
\textit{stability limit / spinodal line}.

How to uncover all negative minima of $\mathit{tm}_z(\vektor{W})$ (if any)?
This cannot be done with total certainty, but one may use some clever initial
estimates to perform a finite set of searches which are likely to uncover
them. 

\subsection{Tangent plane stability analysis}

Tangent plane stability analysis serves two purposes:
\begin{itemize}
  \item Verifies that a single phase state is stable.
  \item If the single phase state is not stable, gives an improved K-factor guess.
\end{itemize}
As shown, a phase of composition $\vektor{z}$ is stable at the specified temperature and pressure \textit{if and only if}
$tpd_z(\vektor{w}) \geq 0$ for any trial phase composition $\vektor{w}$. This could checked by inspecting the minima of 
$\mathit{tm}_z(\vektor{W})$ \eqref{eq:tm} and checking the criterion
$W_T^\text{SP} \leq 1$. 

The original composition is \textit{intrinsically unstable} if the Hessian 
\begin{equation}
  H_{ij} \equiv \frac{\partial^2 \mathit{tm}_z}{\partial W_i \partial W_j} 
  = \frac{\delta_{ij}}{W_i} + \frac{\partial \ln \phi_i}{\partial W_j}
  \label{}
\end{equation}
has any negative eigenvalues at the trivial stationary point at $\vektor{W}=\vektor{z}$. If it has no negative eigenvalues at 
this point, the mixture is either stable ($\mathit{tm}_z(\vektor{W}) > 0$ at all its minima) or 
\textit{metastable} ($\mathit{tm}_z(\vektor{W}) < 0$ at some minimum where $\vektor{W}\neq \vektor{z}$).

By using \eqref{eq:tm_stationary}, minima of $\mathit{tm}_z(\vektor{W})$ may be found by successive substitution:
\begin{equation}
  \ln W_i^{k+1} = d_i(\vektor{z}) - \ln \phi_i \left( \vektor{W}^k \right),
  \label{}
\end{equation}
which may be shown to only converge to minima, not other stationary points. Slow convergence may be remedied by higher order 
methods.
Outcomes of stability analysis:
\begin{itemize}
  \item A composition $\vektor{W}$ where $\mathit{tm}_z(\vektor{W}) < 0$ is found. Continue the flash calculation with $\vektor{w}$ 
    as a new phase.
  \item All attempts converge to either the trivial solution of a positive minimum. The composition $\vektor{z}$ is thus stable.
\end{itemize}

\subsection{Mixture reduced molar Gibbs energy}
By combining \eqref{eq:G_alt}, \eqref{eq:fugcoeff_vs_mu} and \eqref{eq:mu_and_xi_idealmix}, one may show that 
the Gibbs energy of a multi-phase mixture is given by 
\begin{align}
  G &= \sum_j \sum_i n_i^{(j)} \mu_i^{(j)} \nonumber\\
  &= \sum_j\sum_i n_i^{(j)} \left[ \mu_i^\text{ig}(T,P,\vektor{n}) + RT\ln \phi_i^{(j)}(T,P,\vektor{n})\right] \nonumber\\
  &= \sum_j\sum_i n_i^{(j)} \left[ \mu_i^\text{ig}(T,P) + RT\ln x_i^{(j)} + RT\ln \phi_i^{(j)}(T,P,\vektor{n})\right] \nonumber\\
  &= \sum_j\sum_i n_i^{(j)} \mu_i^\text{ig}(T,P) +  \sum_j \sum_i n_i^{(j)} \left[ RT\ln x_i^{(j)} + RT\ln \phi_i^{(j)}(T,P,\vektor{n})\right] \nonumber\\
  &= \sum_i \left( \sum_j n_i^{(j)} \right) \mu_i^\text{ig}(T,P) +  
      \sum_j \sum_i n_i^{(j)} \left[ RT\ln x_i^{(j)} + RT\ln \phi_i^{(j)}(T,P,\vektor{n})\right] \nonumber\\
  &= \sum_i  n_i^{(\text{tot})}  \mu_i^\text{ig}(T,P) +  
      \sum_j \sum_i n_i^{(j)} \left[ RT\ln x_i^{(j)} + RT\ln \phi_i^{(j)}(T,P,\vektor{n})\right] 
  \label{eq:G_mix}
\end{align}

When wanting to compare the total Gibbs energy of different phase configurations possible from the same overall composition, 
the first term in \eqref{eq:G_mix} will cancel due to being identical in all configurations. It is thus not necessary to consider 
it further for these purposes. Denoting this term by $G_0 \equiv \sum_i  n_i^{(\text{tot})}  \mu_i^\text{ig}(T,P)$, one may 
define a reduced molar Gibbs energy useful for comparing different phase configurations from the same feed by 
\begin{align}
  g &\equiv \frac{G - G_0 }{nRT} \nonumber\\
    &= \sum_i \sum_j \beta_j  x_i^{(j)} \left[ \ln x_i^{(j)} + \ln \phi_i^{(j)}(T,P,\vektor{n})\right],
  \label{eq:g_func}
\end{align}
where $\beta_j$ is the fraction of moles in phase $j$.


\subsection{Two-phase flash}
Consider a mixture of $C$ components, of overall composition $\vektor{z}$, and define the liquid and vapor compositions as 
$\vektor{x}$ and $\vektor{y}$, respectively. When indices are used instead of vectors, the index $i$ indicates the component.
It is assumed that we have a thermodynamic model for the mixture, capable of calculating the fugacity of any component, given a set of   
temperature, pressure and composition.

The equilibrium condition is 
\begin{equation}
  f_i^l = f_i^v \quad \forall \quad i
  \label{eq:eq_twophase}
\end{equation}

Define $\beta$ as the overall mole fraction of the vapor phase, giving the material balance:
\begin{equation}
  \beta y_i + (1-\beta)x_i = z_i \quad \forall \quad i
  \label{eq:materialbalance}
\end{equation}

The mole fractions must sum to unity, a condition which may be transformed to the convenient form:
\begin{equation}
  \sum_{i=1}^C \left( x_i - y_i \right) = 0.
  \label{eq:mole_frac_sum}
\end{equation}
The above equations yield $2C + 1$ relations, while the number of variables ($\vektor{x}$, $\vektor{y}$, $T$, $P$, $\beta$) are $2C + 3$. 
This shows that two additional pieces of information are needed, such as specifying $(T,P)$.


\textit{Equilibrium factors}:
\begin{equation}
  K_i \equiv \frac{y_i}{x_i}
  \label{eq:K_factors}
\end{equation}


Combining \eqref{eq:K_factors} and \eqref{eq:materialbalance}
\begin{equation}
  x_i = \frac{z_i}{1 - \beta + \beta K_i}, \quad\quad y_i = K_i x_i = \frac{K_i z_i}{1 - \beta + \beta K_i} 
  \label{eq:xy_and_Kbeta}
\end{equation}

At equilibrium, where \eqref{eq:eq_twophase} is satisfied, we have that 
\begin{equation}
  K_i \equiv \frac{y_i}{x_i} = \frac{\phi_i^l}{\phi_i^v}
  \label{eq:K_eq}
\end{equation}

Combining \eqref{eq:mole_frac_sum} with \eqref{eq:xy_and_Kbeta} leads to the \textit{Rachford-Rice} equation:
\begin{equation}
  g(\beta) \equiv \sum_{i=1}^C \left( x_i - y_i \right) = \sum_{i=1}^C \frac{z_i \left( K_i -1 \right)}{1 -\beta + \beta K_i} = 0
  \label{eq:rachford_rice}
\end{equation}
\begin{framed}
Two-phase TP flash by successive substitution:  
\begin{enumerate}
  \item Use current (or initial) $\vektor{K}$ in the Rachford-Rice equation \eqref{eq:rachford_rice}.
  \item Solve the Rachford-Rice equation for a value of $\beta$ \textit{e.g.} by Newton's method.
  \item Find compositions $x_i$ and $y_i$ corresponding to the above values of $K_i$ and $\beta$, through \eqref{eq:xy_and_Kbeta}.
  \item Find fugacity coefficients from the equation of state given the above compositions, using \textit{e.g.} \eqref{eq:fugcoeff}.
  \item Find a new equilibrium factor $\vektor{K}$ using the above fugacity coefficients and the equilibrium assunption \eqref{eq:K_eq}.
  \item Go to 1 if $\vektor{K}$ has not converged yet.
\end{enumerate}
\end{framed}
Equal fugacities may also be satisfied at ``false solutions'', which are critical points of the Gibbs energy, but not minima.
Fortunately, it may be shown that successive substitution can only converge to minima of the Gibbs energy~\cite{michelsen07}.
Convergence will be slow at high pressures or close to the mixture critical point.

Initial values for $\vektor{K}$:
Wilson's approximation:
\begin{equation}
  \ln K_i = \ln \left( \frac{P_{c_i}}{P} \right) + 5.373(1+\gamma_i) \left( 1-\frac{T_{c_i}}{T} \right),
  \label{eq:wilson_K}
\end{equation}
where $P_{c_i}$, $T_{c_i}$ and $\gamma_{i}$ is the critical temperature, critical pressure and acentric factor, respectively, 
of component $i$.

Single phase solution?\\
Convergence at $\beta <0$ or $\beta>0$, or convergence to the trivial solution $\vektor{x} = \vektor{y}$.\\

Flash strategy:
\begin{enumerate}
  \item Find Wilson K-factors \eqref{eq:wilson_K} to use as initial estimates.
  \item Do three steps of successive substitution. Check for conditions:
    \begin{itemize}
      \item Total Gibbs energy of the resulting liquid-vapor system is lower than in the feed (assuming single phase). This verifies the presence of 
              multiple phases. Continue iteration, perhaps with higher order methods.
      \item If not the above, check  if $\mathit{tpd}_z(\vektor{x})$ or $\mathit{tpd}_z(\vektor{y})$ is negative. If the former, set 
        $K_i = \phi_i^x / \phi_i^z$. If the latter, set $K_i = \phi_i^z / \phi_i^y$. Continue iteration.
    \end{itemize}
  \item If neither of the above conditions were met after three iterations, do tangent plane stability analysis.
\end{enumerate}
If at any time $\beta$ exceeds its bounds, it is very likely that the specification corresponds to a single phase state. 
To be certain, perform tangent plane stability analysis.


Initializing $\vektor{W}$ for tangent plane analysis: 
Investigate both ``liquid-like'' and ``vapor-like'' trial compositions, through 
Wilson K-factors \eqref{eq:wilson_K}. If the feed itself may with confidence be said to be ``liquid-like'' or ``vapor-like'', 
one may to save time only investigate the trial composition of the opposite kind.



\subsubsection{Solution by minimization}
When formulating an equilibrium calculation as a minimization problem, it 
may be useful to use \textit{molar amounts} as the independent variables, 
defined by 
\begin{align}
  v_i &\equiv \beta y_i\nonumber\\
  l_i &\equiv (1-\beta) x_i,
  \label{}
\end{align}
such that
\begin{equation}
  v_i + l_i = z_i,
  \label{}
\end{equation}
where $\vektor{z}$ is now molar amounts, but numerically identical to the feed composition.

\begin{equation}
  y_i = \frac{v_i}{\sum v_i}, \quad x_i = \frac{l_i}{\sum l_i}
  \label{}
\end{equation}

\begin{equation}
  \beta_L \equiv 1-\beta
  \label{}
\end{equation}

\begin{equation}
  \beta = \sum v_i
  \label{}
\end{equation}

\begin{equation}
  \beta_L = \sum l_i
  \label{}
\end{equation}

The minimization problem to be solved is then 
\begin{equation}
  \vektor{v} = \underset{\vektor{v} \in \mathbb{R}^C}{\argmin}\left\{g(\vektor{v},\vektor{l}(\vektor{v}))\right\}, \quad\text{with } 
  \vektor{l} = \vektor{z}-\vektor{v}
  \label{}
\end{equation}
\textit{e.g.}\ by Newton's method of optimization (ref appendix).


By using \eqref{eq:g_func} for the case of two-phase systems:
\begin{equation}
  g = \sum_i \left( v_i \left[ \ln y_i + \ln \phi_i^v \right] + l_i \left[ \ln x_i + \ln \phi_i^l \right] \right)
  \label{eq:g_twophase}
\end{equation}

\begin{align}
  \frac{\partial g}{\partial v_j}
  =& \ln y_j - \ln x_j + \ln \phi_j^v - \ln \phi_j^l \nonumber\\
  &+ 1 - \underbrace{\sum_i y_i}_{=1} + \underbrace{\sum_i v_i \frac{\partial \ln \phi_i^v}{\partial v_j}}_{=0}\nonumber\\
  &- 1 + \underbrace{\sum_i x_i}_{=1} - \underbrace{\sum_i l_i \frac{\partial \ln \phi_i^l}{\partial l_j}}_{=0} \nonumber\\
  =& \ln y_j - \ln x_j + \underbrace{\ln \phi_j^v - \ln \phi_j^l}_{\text{From EoS}},
  \label{}
\end{align}
where the sums involving derivatives are zero due to \textit{Euler's theroem} for homogeneous functions, as explained in
1.8 of~\cite{michelsen07}.

\begin{align}
  \frac{\partial^2 g}{\partial v_i \partial v_j} 
  &= \frac{\partial \ln \phi_j^v}{\partial v_i} - \frac{\partial \ln \phi_j^l}{\partial v_i}  
  - \left( \frac{\beta + \beta_L}{\beta \beta_L} \right)
  + \delta_{i,j} \left( \frac{z_i}{x_i y_i \beta \beta_L} \right) \nonumber\\
  &= 
  \frac{1}{\beta \beta_L} \left[
  \beta_L \underbrace{\left( \sum_i v_i\right)  \frac{\partial \ln \phi_j^v}{\partial v_i} }_\text{From EoS}
  +  \beta \underbrace{\left( \sum_i l_i\right)  \frac{\partial \ln \phi_j^l}{\partial l_i}}_\text{From EoS}
  - 1
  + \delta_{i,j} \left( \frac{z_i}{x_i y_i} \right) 
  \right]\nonumber\\
  \label{}
\end{align}

The function $g$ in \eqref{eq:g_twophase} should at the located minimum be smaller than the smallest possible value 
for the a single phase of feed composition:
\begin{equation}
  g_\text{feed} = \min\left\{ \sum_i  z_i \left[ \ln z_i + \ln \phi_i(\vektor{z}) \right]   \right\},
  \label{}
\end{equation}
which is either a liquid or a vapor state, depending on which has the lowest Gibbs energy.


\section{Cubic equations of state}
The general cubic equation of state has the form
\begin{equation}
P = \frac{RT}{v -b}- \frac{\alpha a}{(v-m_1 b)(v-m_2 b)},
\label{eq:gencubic}
\end{equation}
where $m_1$ and $m_2$ are dimensionless constants defining the SRK, PR and Van der
Waals equation of state, $v=V/n~(\SI{}{\meter^3\per\mol})$ is the specific volume. The
parameters $a~(\SI{}{\meter^3\joule\per \mol^2})$ and $b~(\SI{}{\meter^3\per\mol})$ typically 
depend on the composition and the dimensionless quantity 
$\alpha$ is a function of temperature. 



\begin{shaded}
  \noindent
\textit{Exercise: Find the reduced residual Helmholtz energy function for a general
cubic equation of state \eqref{eq:gencubic}.}

\noindent
\textit{Solution}:
\begin{equation}
  A^\text{r}(T,V,N) = n \left[ RT\ln \left( \frac{v}{v-b} \right) + 
  \frac{\alpha a}{\left( m_1-m_2 \right)b} \ln\left( \frac{v-m_1 b}{v-m_2 b} \right) \right]
  \label{}
\end{equation}

\end{shaded}

\subsection{Mixing rules}
Mixing rules are various ways of letting $a$ and $b$ in \eqref{eq:gencubic} 
depend on composition. 

\subsubsection{Classic mixing rule}
Parameters: $a_{ii}$, $k_{ij}$ and $b_i$.
\begin{equation}
  a = \sum_i x_i \sum_j x_j a_{ij}
  \label{}
\end{equation}

\begin{equation}
  a_{ij} = a_{ji} = \sqrt{a_{ii} a_{jj}} \left( 1-k_{ij} \right)
  \label{}
\end{equation}

\begin{equation}
  b = \sum_i x_i b_i
  \label{}
\end{equation}


\subsubsection{Huron-Vidal mixing rule}



\section{Explicit Helmholtz equations of state}
Based on an explicit expression for the reduced dimensionless Helmholtz energy \eqref{eq:reduced_helmholtz}. This includes 
equations such as GERG2004/2008 and the Span-Wagner equation of state for \ce{CO2}.

\begin{equation}
  a(\delta,\tau,\vektor{x}) = a^\text{ig}(\rho,T,\vektor{x}) + a^\text{r}(\delta,\tau,\vektor{x}),
  \label{}
\end{equation}
where $\delta$ is the \textit{reduced mixture density} and $\tau$ is the \textit{reduced mixture temperature}, which are both generally 
dependent on the composition:
\begin{equation}
  \delta = \frac{\rho}{\rho_r(\vektor{x})}, \quad \quad \tau = \frac{T}{T_r(\vektor{x})}.
  \label{}
\end{equation}


\section{Ideal gas Helmholtz energy}
The equation of state does not provide the energies.
In order to find the total Helmholtz energy, and not just the residual, 
the ideal gas contribution $A^\text{ig}$ must be found.

An ideal gas is defined by two equations. One the equation of state
\begin{equation}
  P^\text{ig} = \frac{nRT}{V}
  \label{eq:idealgaslaw}
\end{equation}
and the second is the equation for the internal energy
\begin{equation}
  U^\text{ig}(T,n) = n c_v(T) T
  \label{eq:U_ideal}
\end{equation}
which simply states that the internal energy depends on the temperature and the number of particles only. The function $c_v$
is the \textit{heat capacity at constant volume}: 
\begin{equation}
  C_v = n c_v \equiv \pone{Q}{T}{V} = \pone{U}{T}{V}.
  \label{}
\end{equation}
Additionally there is the \textit{heat capacity at constant pressure}:
\begin{equation}
  C_p = n c_p \equiv \pone{Q}{T}{P} = \pone{H}{T}{P},
  \label{}
\end{equation}
where it was used that $\dd{H} = \delta Q + V\dd{P}$.
For an ideal gas, the enthalpy is 
\begin{align}
  H^\text{ig}(T,n) &\equiv U^\text{ig} + p^\text{ig}V 
  \nonumber\\
  &= U^\text{ig}(T,n) + nRT
  \nonumber\\
  &= n\left( c_v + R \right)T
  \label{}
\end{align}
which shows that for an ideal gas
\begin{equation}
  c_p = c_v + R.
  \label{}
\end{equation}

Change in ideal gas enthalpy at a constant number of particles:
\begin{align}
  H^\text{ig}(T,n) - H^\text{ig}_0 &= \int_{T_0}^T \pone{H}{T}{P} \dd{T}
  \nonumber\\
   &= n \int_{T_0}^T c_p \dd{T}
  \label{}
\end{align}

Change in ideal gas entropy at a constant number of particles:
\begin{align}
  S^\text{ig}(T,V,n) - S^\text{ig}_0 
  &= \int_{T_0}^T \pone{S}{T}{V} \dd{T} + \int_{V_0}^V \pone{S}{V}{T} \dd{V}
    \nonumber\\
    &=\int_{T_0}^T \frac{n c_v(T)}{T} \dd{T}
    + \int_{V_0}^V \pone{P}{T}{V} \dd{V}
    \nonumber\\
    &=n\int_{T_0}^T \frac{c_p(T) - R}{T} \dd{T}
    + nR\int_{V_0}^V \frac{1}{V} \dd{V}
    \nonumber\\
    &=n\int_{T_0}^T \frac{c_p(T) - R}{T} \dd{T}
    + nR\ln \left( \frac{V}{V_0} \right),
  \label{}
\end{align}
using 
\begin{equation}
  \pone{S}{T}{V} = \pone{S}{U}{V} \pone{U}{T}{V} = \frac{1}{T} \pone{U}{T}{V} = \frac{n c_v}{T} 
  \label{}
\end{equation}
and the \textit{Maxwell relation}
\begin{equation}
\label{eq:dsrel}
\pone{S}{V}{T,\vektor{n}} = \pone{P}{T}{V,\vektor{n}}.
\end{equation}

The total Helmholtz energy is not provided through the cubic equation of state, only the residual part.
The ideal gas part is usually specified as 
algebraic temperature functions for the heat capacity at constant pressure, $c_p(T)$, provided
for each component in the mixture. The total Helmholtz energy may be written as 
\begin{align}
  A &\equiv U - TS \nonumber\\
  &=U-TS+PV-PV \nonumber\\
  &= H - TS - PV,
  \label{}
\end{align}
which means that the Helmholtz energy of a pure ideal gas may be expressed as:
\begin{align}
  A^\text{ig}_i(T,V,n_i) &= H^\text{ig} - TS^\text{ig} - nRT
  \nonumber\\
  &= \Delta H_i^\text{ig} + H_{0,i}^\text{ig} 
  - T\left(  \Delta S_i^\text{ig} + S_{0,i}^\text{ig}  \right)  - n_i RT
  \nonumber\\
  &=  n_i \int_{T_0}^T c_{p,i}^\text{ig}(T) \dd{T} + H_{0,i}^\text{ig}
  -  n_i T \left[ \int_{T_0}^T \frac{c_{p,i}^\text{ig}(T) - R}{T} \dd{T} 
    + R \ln \left( \frac{V}{V_0} \right) \right]
     - T S_{0,i}^\text{ig} - nRT.
  \label{}
\end{align}

Mixing carries with it another entropy contribution, even in the case of ideal mixtures, \textit{i.e.} with no interactions. The entropy change 
from ideal mixing at constant pressure and temperature is:
\begin{align}
  \Delta S^\text{ig}_\text{mix} &= -nR \sum_i x_i \ln x_i 
  \nonumber\\
  &= -R \sum_i n_i \ln \left( \frac{n_i}{n} \right),
  \label{}
\end{align}
which means that the total Helmholtz energy of an ideal mixture is given by:
\begin{align}
  A^\text{ig}(T,V,\vektor{n}) &=  \sum_i A_i^\text{ig}(T,V,n_i) - T \Delta S^\text{ig}_\text{mix}
  \label{}
\end{align}
In terms of the reduced Helmholtz energy, this becomes:
\begin{align}
  a^\text{ig}(T,V,\vektor{x}) &=  \sum_i x_i a_i^\text{ig}(T,V) + \sum_i x_i \ln x_i, 
  \label{}
\end{align}
where 
\begin{equation}
  a_i^\text{ig}(T,V) \equiv \frac{A^\text{ig}_i(T,V,n_i)}{n_i RT}.
  \label{}
\end{equation}


\appendix

\section{Newton's method for root finding}
\section{Newton's method for optimization (Modified Newton)}
\section{Accelerated direct substitution (Dominant Eigenvalue Method)}


\begin{thebibliography}{9}

\bibitem{callen85}
  Herbert B. Callen,
  \emph{Thermodynamics and an Introduction to Thermostatistics}.
  second edition,
  John Wiley \& Sons Inc,
  1985.

\bibitem{kunz07}
  O. Kunz, R. Klimeck, W. Wagner, M. Jaeschke,
  \emph{The GERG-2004: Wide-Range Equation of State for Natural Gases and Other Mixtures}.
  GERG TM15 Report (http://www.gerg.eu/publications/tm.htm),
  2007.

\bibitem{michelsen82a}
  Michael L. Michelsen,
  \emph{The isothermal flash problem. Part I. stability}.
  Fluid Phase Equilibria,
  9(1)
  pp. 1-19,
  1982.

\bibitem{michelsen82b}
  Michael L. Michelsen,
  \emph{The isothermal flash problem. Part II. phase-split calculation}.
  Fluid Phase Equilibria,
  9(1)
  pp. 21-40,
  1982.

\bibitem{michelsen99}
  Michael L. Michelsen and J{\o}rgen M. Mollerup,
  \emph{State function based flash specifications}.
  Fluid Phase Equilibria,
  158-160
  pp. 617-626,
  1999.

\bibitem{michelsen07}
  Michael L. Michelsen and J{\o}rgen M. Mollerup,
  \emph{Thermodynamic models: Fundamentals \& computational aspects}.
  second edition
  Tie-Line Publication,
  2007.

\bibitem{span96}
  Roland Span and Wolfgang Wagner,
  \emph{A New Equation of State for Carbon Dioxide Covering the Fluid Region
  from the Triple-Point Temperature to 1100 {K} at Pressures up to 800 {MPa}}.
  J. Phys. Chem. Ref. Data,
  25(6)
  pp. 1509-1596
  1996.

\end{thebibliography}

\end{document}
